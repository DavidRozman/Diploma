\documentclass[12pt,a4paper]{amsart}
% ukazi za delo s slovenscino -- izberi kodiranje, ki ti ustreza
\usepackage[slovene]{babel}
%\usepackage[cp1250]{inputenc}
%\usepackage[T1]{fontenc}
\usepackage[utf8]{inputenc}
\usepackage{amsmath,amssymb,amsfonts,mathtools}
\usepackage{url}
%\usepackage[normalem]{ulem}
\usepackage[dvipsnames,usenames]{color}
\usepackage{biblatex}
\addbibresource{viri.bib}
\nocite{*}
\usepackage{pdfpages}
\usepackage{graphicx}
\usepackage{algorithm}
\usepackage{algpseudocode}
\usepackage{listings}
\usepackage{color} %red, green, blue, yellow, cyan, magenta, black, white
\definecolor{mygreen}{RGB}{28,172,0} % color values Red, Green, Blue
\definecolor{mylilas}{RGB}{170,55,241}
\usepackage{float}

% ne spreminjaj podatkov, ki vplivajo na obliko strani
\textwidth 15cm
\textheight 24cm
\oddsidemargin.5cm
\evensidemargin.5cm
\topmargin-5mm
\addtolength{\footskip}{10pt}
\pagestyle{plain}
\overfullrule=15pt % oznaci predlogo vrstico


% ukazi za matematicna okolja
\theoremstyle{definition} % tekst napisan pokoncno
\newtheorem{definicija}{Definicija}[section]
\newtheorem{primer}[definicija]{Primer}
\newtheorem{opomba}[definicija]{Opomba}
\newtheorem{zgled}[definicija]{Zgled}

\renewcommand\endprimer{\hfill$\diamondsuit$}


\theoremstyle{plain} % tekst napisan posevno
\newtheorem{lema}[definicija]{Lema}
\newtheorem{izrek}[definicija]{Izrek}
\newtheorem{trditev}[definicija]{Trditev}
\newtheorem{posledica}[definicija]{Posledica}


% za stevilske mnozice uporabi naslednje simbole
\newcommand{\R}{\mathbb R}
\newcommand{\N}{\mathbb N}
\newcommand{\Z}{\mathbb Z}
\newcommand{\C}{\mathbb C}
\newcommand{\Q}{\mathbb Q}
\newcommand{\okr}{\text{exp}_{\lambda}}
\newcommand{\ud}{\mathrm{d}}


% ukaz za slovarsko geslo
\newlength{\odstavek}
\setlength{\odstavek}{\parindent}
\newcommand{\geslo}[2]{\noindent\textbf{#1}\hspace*{3mm}\hangindent=\parindent\hangafter=1 #2}

% naslednje ukaze ustrezno popravi
\newcommand{\program}{Finančna matematika} % ime studijskega programa: Matematika/Finan"cna matematika
\newcommand{\imeavtorja}{David Rozman} % ime avtorja
\newcommand{\imementorja}{doc.~dr. Uroš Kuzman} % akademski naziv in ime mentorja
\newcommand{\naslovdela}{Diferencialne enačbe z zamikom}
\newcommand{\letnica}{2022} %letnica diplome


% vstavi svoje definicije ...




\begin{document}

% od tod do povzetka ne spreminjaj nicesar
\thispagestyle{empty}
\noindent{\large
UNIVERZA V LJUBLJANI\\[1mm]
FAKULTETA ZA MATEMATIKO IN FIZIKO\\[5mm]
\program\ -- 1.~stopnja}
\vfill

\begin{center}{\large
\imeavtorja\\[2mm]
{\bf \naslovdela}\\[10mm]
Delo diplomskega seminarja\\[1cm]
Mentor: \imementorja}
\end{center}
\vfill

\noindent{\large
Ljubljana, \letnica}
\pagebreak

\thispagestyle{empty}
\tableofcontents
\pagebreak

\thispagestyle{empty}
\begin{center}
{\bf \naslovdela}\\[3mm]
{\sc Povzetek}
\end{center}
% tekst povzetka v slovenscini
 %V povzetku na kratko opi"site vsebinske rezultate dela. Sem ne sodi razlaga organizacije dela -- v katerem poglavju/razdelku je kaj, pa"c pa le opis vsebine.
Diferencialne enačbe z zamikom povezujejo odvod funkcije z njenimi vrednosti v preteklem času. Začetni pogoj je 
podan kot funkcija na ustreznem intervalu. Ker je reševanje takih enačb še bolj zahtevno kot reševanje navadnih 
diferencialnih enačb, je zanje značilna uporaba numeričnih metod. Pogoji za obstoj in enoličnost rešitve so podobni kot
pri navadnih diferencialnih enačbah, vendar pa rešitve pogosto niso zvezne in odvedlijve. V nalogi je predstavljenih 
nekaj standardnih metod za reševanje 
izbranih tipov enačb, analizirana pa sta tudi zamaknjena modela eksponentne in logistične rasti.
 \vfill
\begin{center}
{\bf Delay dif\mbox{}ferential equations}\\[3mm] % prevod slovenskega naslova dela 
{\sc Abstract}
\end{center}
% tekst povzetka v anglescini
Delay differential equations connect the derivative of a function with its value in a previous state. The initial 
condition is given as a function over a certain interval. Solving such equations is generally much harder than solving 
ordinary dif\mbox{}ferential equations, therefore numerical methods are commonly used in the process. The conditions for 
existence and uniqueness of a solution are similar to those for ordinary dif\mbox{}ferential equations, yet the solution
is often neither dif\mbox{}ferentiable nor continuous. In this thesis I present some of the standard methods for solving certain 
types of delay dif\mbox{}ferential equations and analyse delayed exponential and logistic growth models.

\vfill\noindent
{\bf Math. Subj. Class. (2010):} 34K05, 34K06, 34K20  \\[1mm]  
{\bf Klju"cne besede:} Diferencialne enačbe, zamik, eksistenčni izrek, Laplaceova transformacija, logistična funkcija  \\[1mm]  
{\bf Keywords:} Dif\mbox{}ferential equations, delay, existence theorem, Laplace transform, logistic function
\pagebreak



% tu se zacne tekst seminarja
\section{Uvod}

Diferencialne enačbe spadajo med najpomembnejša področja matematične analize. So enačbe funkcije ene ali več
spremenljivk, ki povezujejo njene vrednosti z njenimi odvodi. Poleg matematike so ključne pri modeliranju tudi v drugih
znanstvenih vedah, npr. v fiziki, kemiji, astronomiji in ekonomiji.

Začetki diferencialnih enačb segajo v drugo polovico 17. stoletja, ko sta Isaac Newton in Gottfried Wilhelm von 
Leibniz predstavila infinitizemalni račun. V naslednjih sto letih so veliko dela preučevanju diferencialnih
enačb namenili največji svetovni matematiki, kot so Euler, Bernoulli in Lagrange, ter odkrili nekaj najpomembnejših
metod za reševanje.

Kljub mnogim raziskavam diferencialne enačbe še danes ostajajo kompleksen matematični problem. Podobno kot pri
nedoločenem integralu, splošni postopek za njihovo reševanje ne obstaja. Ker znamo reševati le določene tipe in je
eksplicitno iskanje rešitve v praksi redko izvedljivo, za reševanje pogosto uporabimo numerične metode,
pogosto pa se omejimo tudi zgolj na obstoj rešitve.

Navadne diferencialne enačbe povezujejo odvod funkcije v neki točki z njenimi trenutnimi vrednostmi.
V tej nalogi bomo obravnavali diferencialne enačbe z zamikom, pri katerih je vrednost odvoda iskane funkcije odvisna
tudi od njenih preteklih vrednosti. Motivacija za njihov študij je bila želja po bolj realističnih modelih za procese, 
katerih trenutno stanje je odvisno tudi od njihovih stanj v preteklosti. Take enačbe se zato pogosto pojavijo v modelih
iz biologije, kemije in mehanike.

Na začetku naloge bomo na kratko ponovili osnove navadnih diferencialnih enačb. Nato bomo definirali
diferencialne enačbe z zamikom in si na preprostem zgledu pogledali primer reševanja in rešitev primerjali z 
rešitvijo analogne navadno diferencialne enačbe. V podrobnejši analizi bomo navedli eksistenčne izreke, torej kdaj 
rešitev obstaja ter kakšne lastnosti ima iskana funkcija. Za tem si bomo pogledali nekaj osnovnih tipov enačb z 
njihovimi postopki za reševanje. Na koncu pa bomo predstavili še nekaj zgledov uporabe takih enačb pri modeliranju.


\section{Primerjava NDE in DDE}

Za začetek se spomnimo osnov navadnih diferencialnih enačb.

\begin{definicija}
    Navadna diferencialna enačba (NDE) reda $n$ je enačba oblike
    \[F(x,y,y',y'',\dots,y^{(n)})=0,\]
    kjer je $F$ dana funkcija $n+2$ spremenljivk, $y$ neznana funkcija spremenljivke $x$, $y^{(k)}$ pa njeni
    odvodi.
\end{definicija}
Red enačbe je torej red najvišjega odvoda, ki nastopa v enačbi. Pridevnik \textit{navadna} se nanaša na dejstvo,
da v njej nastopa funkcija ene spremenljivke. V splošnem poznamo 
tudi parcialne diferencialne enačbe za funkcije več spremenljivk, ki pa jih v tej nalogi ne bomo obravnavali.

Oglejmo si primer zelo preproste NDE z ločljivimi spremenljivkami.

\begin{zgled}
    Podana je NDE z začetnim pogojem:
    \[\dot{x}(t) =-x(t),\quad x(0)=1.\]
    Rešimo jo s standardnim postopkom za enačbe z ločljivimi spremenljivkami:
    \begin{equation*}
    \begin{split}
      \frac{dx}{dt} & =-x \\
      \frac{dx}{-x} & =dt \\
      \int \frac{dx}{-x} & =\int dt \\
     -\ln|x| & =t+c \\
      x(t) & =De^{-t} \\
    \end{split}
    \end{equation*}
Ko upoštevamo še začetni pogoj, dobimo $D=1$. Končna rešitev je torej \[x(t)=e^{-t}.\]
Rešitev je funkcija, ki je zvezna in odvedljiva na realni osi.
\end{zgled}


\begin{figure}[h]
    \includegraphics[width=10cm, height=6cm]{exp.jpg}
    \caption{Graf funkcije $x(t)=e^{-t}$}
\end{figure}


Pri NDE opazimo dve pomembni lastnosti, ki v nadaljevanju pri diferencialnih enačbah z zamikom ne bosta veljali. 
Vrednost odvoda funkcije $x$ v točki $t$ je bila odvisna le od njene vrednosti v tej isti točki. Poleg tega smo 
imeli še en začetni pogoj, zaradi katerega smo dobili enolično rešitev. Ta pogoj je bil podan kot vrednost funkcije
$x$ v eni točki. Običajna verzija eksistenčnega izreka pove, da je rešitev tovrstnega začetnega problema
ena sama, ker desna stran izpolnjuje Lipschitzev pogoj. Splošneje pa vemo, da
je splošna rešitev NDE reda $n$, če obstaja, $n$-parametrična.
Z vsakim dodatnim začetnim pogojem, ki je podan kot vrednost funkcije ali enega izmed njenih odvodov v fiksni točki,
se število parametrov v splošni rešitvi zmanjša za ena.

Če NDE iz primera zapišemo kot $(\dot{t},\dot{x})=(1,-x)$,
lahko narišemo tudi njen fazni portret.
\begin{figure}[h]
    \includegraphics[width=10cm, height=10cm]{tempsnip.png}
    \caption{Fazni portet NDE $\dot{x} = -x$}
\end{figure}


\newpage

Sedaj uvedimo osrednji pojem diplomske naloge, torej diferencialne enačbe, v 
katerih odvod funkcije ene spremenljivke ne bo odvisen samo od vrednosti funkcije
v trenutnem času ampak tudi v neki pretekli točki. 

\begin{definicija}
    Naj bo $\tau > 0$ konstanta in $J = [\xi,\xi +a]$, kjer je $\xi \geq 0$ in $a > 0$. Enačbi oblike
    \[\dot{x}(t)=f(t,x(t-\tau))\, \quad \text{za } t \in J\]
    pravimo diferencialna enačba prvega reda z zamikom.
\end{definicija}

\noindent V nadaljevanju bomo take enačbe
označevali s kratico DDE, kar ustreza angleškemu prevodu njenega imena - 
\textit{delay dif\mbox{}ferential equation}. Konstanti $\tau>0$ pravimo zamik.
Začetni pogoj za tovrstne enačbe je podan s predpisom
\[x(t)=\phi(t) \quad \text{za } t \in J_{-} = [\xi-\tau,\xi],\]kjer je $\phi$ zvezna funkcija.

\textit{Opomba:} V nadaljevanju diplomske naloge, bo izraz začetni problem pomenil diferencialno enačbo z
zamikom skupaj z začetnim pogojem zanjo.

Pomembna razlika v primerjavi z začetnimi problemi, ki smo jih obravnavali pri teoriji za NDE, je 
dejstvo, da začetni pogoj ni več podan kot vrednost 
funkcije v eni točki, temveč kot neka funkcija na intervalu, ki ponazarja negativne čase. 
Zaradi tega lahko sklepamo, da je družina vseh rešitev DDE neskončno
dimenzionalna.

\begin{zgled}
    Rešimo sledečo DDE:
    \[\dot{x}(t)=-x(t-\tau), \quad \text{za } t \geq 0,\]
    pri začetnem pogoju \[x(t)=1, \quad t \in [-\tau,0].\]
    Najprej rešimo za $t \in [0,\tau]$. Tedaj je $t-\tau \in [-\tau,0]$, zato je
    \[\dot{x}(t)=-x(t-\tau)=-1.\]
    Iz tega sledi
    \[x(t)=x(0)+ \int_{0}^{t}(-1)ds = 1-t, \quad \text{za } t \in [0,\tau].\]
    Tako smo dobili rešitev za interval $[0,\tau]$
    Analogno nadaljujemo za $t \in [\tau, 2\tau]$. Tu velja $t-\tau \in [0,\tau]$, torej
    \[\dot{x}(t)=-x(t-\tau)=-(1-(t-\tau))=(t-\tau)-1.\]
    Od tod sledi
    \[x(t)=x(\tau)+\int_{\tau}^{t}((s-\tau)-1)ds=1-t+\frac{(t-\tau)^2}{2},\quad \text{za }t\in[\tau,2\tau].\]
    Dokažimo sedaj, da za vsak interval $[(n-1)\tau,n\tau], n\in \N$ velja
    \[x(t)=1+\sum_{k=1}^{n}(-1)^k\frac{(t-(k-1)\tau)^k}{k!}.\]
    Za $n=1$ dobimo: $x(t)=1-t$, kar je v skladu z že izračunano rešitvijo.
    Predpostavimo sedaj, da formula drži za nek $n \in \N$ in dokažimo, da potem velja tudi za $n+1$.
    Ker je $t \in [n\tau,(n+1)\tau]$, je $t-\tau \in [(n-1)\tau, n\tau]$. Zato je:
    \[ \dot{x}(t)=-x(t-\tau)=-\left(1+\sum_{k=1}^{n}(-1)^k\frac{(t-\tau-(k-1)\tau)^k}{k!}\right).\]
    Iz tega sledi:
    \begin{equation*}
    \begin{split}
        x(t) &= x(n\tau) - \int_{n\tau}^{t}\left(1+\sum_{k=1}^{n}(-1)^k\frac{(s-k\tau)^k}{k!}\right)ds \\
        & = x(n\tau) - \int_{n\tau}^{t}ds - \int_{n\tau}^{t}\sum_{k=1}^{n}(-1)^k\frac{(s-k\tau)^k}{k!}ds \\
        & = x(n\tau) - t + n\tau - \sum_{k=1}^{n}\frac{(-1)^k}{k!}\int_{n\tau}^{t}(s-k\tau)^{k}ds \\
        & = x(n\tau) - t + n\tau - \sum_{k=1}^{n}\frac{(-1)^k}{k!}\left(\frac{(s-k\tau)^{k+1}}{k+1}\Bigr|_{n\tau}^{t}\right) \\
        & = x(n\tau) - t + n\tau - \sum_{k=1}^{n}\frac{(-1)^k}{(k+1)!}\left((t-k\tau)^{k+1}-((n-k)\tau)^{k+1}\right) \\
        & = 1 + \sum_{k=1}^{n}\left((-1)^k\frac{(n\tau-(k-1)\tau)^k}{k!}\right)- t + n\tau - \sum_{k=1}^{n}\frac{(-1^k)}{(k+1)!}
        (t-k\tau)^{k+1} - \\
        &- \sum_{k=1}^{n}\frac{(-1)^{k+1}}{(k+1)!}((n-k)\tau)^{k+1}.
    \end{split}      
    \end{equation*}
    Ko združimo prvo in tretjo vsoto dobimo $-n\tau$. Druga vsota pa skupaj z $-t$ da
     $\sum_{k=1}^{n+1}(-1)^{k}\frac{(t-(k-1)\tau)^k}{k!}$. Končno dobimo:
     \[x(t) = 1+ \sum_{k=1}^{n+1}(-1)^{k}\frac{(t-(k-1)\tau)^k}{k!},\]
    kar je pa ravno predpostavljena formula za interval $[n\tau, (n+1)\tau].$
    Dobljena rešitev je enolična, zvezna na $[-\tau,\infty)$, ni pa odvedljiva v točki $t=0$, saj je 
    levi odvod enak 0, desni pa -1. To nakazuje, da gladka odvisnost odvoda $\dot{x}(t)$ od zamaknjene
    funkcije $x(t-\tau)$
    ne zagotavlja gladkosti rešitve, ki bi jo dobili
    v NDE za $\tau=0$. %Še podrobneje se bomo s tovrstno analizo ukvarjali v naslednjem razdelku.
    \begin{figure}[h]
        \includegraphics[height=8cm, width=14cm]{bolsa.png}
        \caption{Rešitev za $\tau=1$}
    \end{figure}
\end{zgled}
\noindent Opazimo, da začetni pogoj tu veliko bolj vpliva na obliko končne rešitve, saj je podan kot funkcija in 
ne le kot številka. Zato koncept faznega portreta za diferencialne enačbe z zamikom nima smisla.


\section{Obstoj in enoličnost rešitve}

V tem razdelku bomo dokazali eksistenčni izrek
za začetni problem, ki ga porodi osnovni tip diferencialnih enačb z zamikom. Poglavje je povzeto po 
viru \cite{angdiploma}.

\begin{izrek}
    Dan je začetni problem $$\dot{x}(t)=f(t,x(t-\tau)),$$
    $$x(t)=\phi(t),\quad t\in J_{-},$$ kjer je $f$ zvezna funkcija na traku $S=J \times \R$, $\phi$ zvezna na $J_{-}$ in $\tau >0$
    konstanta. Potem obstaja natanko ena rešitev tega začetnega problema.
\end{izrek}

\begin{proof}[Dokaz]
    Naj bo \[\dot{x}_n(t)=f(t,x_{n}(t-\tau)) \text{ za } t \in [\xi + (n-1)\tau,\xi + n\tau],\]
    kjer je $n \in \N$.
    Iz $t \in [\xi, \xi + \tau]$ sledi $t-\tau \in [\xi - \tau, \xi]$. Za rešitev $x_{1}$ mora torej veljati:
    \[\dot{x}_1(t)=f(t,x_{1}(t-\tau))=f(t,\phi(t-\tau)),\quad x_{1}(\xi)=\phi(\xi).\]
    Iz tega sledi, da je
    \begin{equation*}
        \begin{split}
            x_{1}(t) &= x_{1}(\xi) + \int_{\xi}^{t}f(s, x_{1}(s-\tau))ds \\
            & = \phi(\xi) + \int_{\xi}^{t}f(s,\phi(s-\tau))ds.
        \end{split}      
        \end{equation*}
    Funkcija $x_{1}$ je torej enolično definirana na intervalu $[\xi, \xi + \tau]$. Analogno lahko pokažemo, da velja
    \[\dot{x}_2(t)=f(t,x_{2}(t-\tau))=f(t,x_{1}(t-\tau)) \text{ za } t \in [\xi + \tau, \xi + 2\tau]\] in 
    \[x_{2}(\xi + \tau)=x_{1}(\xi + \tau).\]
    Torej velja:
    \begin{equation*}
        \begin{split}
            x_{2}(t) &= x_{2}(\xi + \tau) + \int_{\xi + \tau}^{t}f(s, x_{2}(s-\tau))ds \\
            & = x_{1}(\xi + \tau) + \int_{\xi + \tau}^{t}f(s, x_{1}(s-\tau))ds.
        \end{split}      
        \end{equation*}
    Funkcija $x_{2}$ je torej enolično določena na intervalu $[\xi + \tau,\xi + 2\tau]$. Induktivno sklepamo, da lahko za vsak 
    $n \in \N$ definiramo $x_{n}$ s pomočjo $x_{n-1}$. Torej je $x_{n}$ enolično določena na intervalu 
    $[\xi + (n-1)\tau, \xi + n\tau]$ za vsak $n \in \N$. Povedano drugače, funkcija podana s predpisom
    \[
        x(t) =
        \begin{dcases}
            \phi(t) &\quad \text{za  } t \in J_{-}=[\xi - \tau, \xi] \\
            \phi(\xi) + \int_{\xi}^{t}f(s,x(s-t))ds &\quad \text{za  } t \in J=[\xi, \xi +a] \\
        \end{dcases}
    \]
    je dobro definirana za vse $t \in J_{-} \cup J = [\xi - \tau, \xi + a]$ in je rešitev začetnega problema. Iz 
    postopka pa sledi, da je enolična.
\end{proof}

\section{Model eksponentne rasti z zamikom}


V tem poglavju si bomo pogledali nekoliko splošnejšo verzijo DDE iz 3. poglavja, ki je podana na sledeči način:
\[\dot{x}(t)= -\alpha x(t-\tau), \quad \tau > 0,\quad \alpha \in \R.\]
Poglavje je povzeto po viru \cite{angdiploma}.


Enačba je odvisna od dveh parametrov, zato jo zaradi lažjega reševanja zreduciramo na enega.
Z uvedbo nove spremenljivke $\mu:=\eta t, \eta > 0$, in funkcijo
$U(\mu)=x(t)$ želimo enačbo prevesti na tako, kjer bo zamik $\tau$ enak 1:
\[\frac{dU}{d\mu}=\frac{dx}{\eta dt}=-\alpha \eta^{-1}x(t-\tau)=-\alpha \eta^{-1}U(\eta t - \eta \tau)=
-\alpha \eta^{-1}U(\mu - \eta \tau).\]
Če izberemo $\eta := 1/\tau$ in $\beta:=\alpha \tau$, dobimo:
\begin{equation} \label{eq1}
    \begin{split}
 \frac{dU}{d\mu} & =-\beta U(\mu -1).
    \end{split}
\end{equation}
Na množici odvedljivih funkcij, definiramo sledeči linearni operator:
\[L(U)=\frac{dU}{d\mu} + \beta U(\mu -1).\]
Ker obravnavana DDE spominja na NDE, pri katerih so rešitve eksponentne funkcije, poskusimo tudi tukaj
iskati rešitve DDE \eqref{eq1} z nastavkom $U(\mu)=e^{\lambda \mu}$. Ko ga vstavimo v linearni operator,
dobimo:
\[L(e^{\lambda \mu})= \lambda e^{\lambda \mu} + \beta e^{\lambda (\mu -1)}=
e^{\lambda \mu}(\lambda + \beta e^{-\lambda}).\]
Iščemo $\lambda \in \C$ tako, da bo vrednost dobljenega izraza enaka 0. Ker
za vsako kompleksno število $\lambda$ velja $e^{\lambda \mu} \neq 0$ , moramo rešiti naslednjo karakteristično
enačbo:
\[ h(\lambda) \equiv \lambda + \beta e^{-\lambda} =0.\]
Če je $\lambda$ ničla te enačbe, je $e^{\lambda\mu}$ rešitev DDE $(1)$.
Nastavek $\lambda = a + ib$ vstavimo v enačbo:
\begin{equation*}
    \begin{split}
        a + ib + \beta e^{-(a+ib)} &= 0 \\
        a + ib + \beta e^{-a}e^{-ib} &= 0 \\
        a + ib + \beta e^{-a}(\cos(b) + \sin(-b)) &= 0 \\
        a + \beta e^{-a}\cos(b) + i(b-\beta e^{-a}\sin(b)) &= 0
    \end{split}      
\end{equation*}
Dobimo sistem enačb za $a$ in $b$:
\begin{equation*}
    \begin{split}
        a &= -\beta e^{-a}\cos(b) \\
        b &= \beta e^{-a}\sin(b).
    \end{split}      
\end{equation*}
%Spomnimo se: Za $n$-krat zvezno odvedljivo funkcijo $h$ je $\lambda \in \C$ ničla reda $n$, če velja:
%\[ h(\lambda) = h'(\lambda) = \dots = h^{(n-1)}(\lambda)=0, \quad h^{(n)}(\lambda)\neq 0.\]

\begin{trditev}
    Števila $\mu^{j}e^{\lambda \mu}, j=0,1,\dots,n$, so rešitve enačbe \eqref{eq1} natanko
    tedaj, ko je $\lambda$ $(n+1)$-kratna ničla karakteristične funkcije $h$.
\end{trditev}

\begin{proof}[Dokaz]
    Linearni operator $L$ $n$-krat odvajamo po $\lambda$. Ker $k$-ti odvod komutira z $L$, dobimo:
    \[L(\mu^{k}e^{\lambda\mu})=\frac{\partial^k}{\partial\lambda^k}(e^{\lambda\mu}h(\lambda))
    =e^{\lambda\mu}\left(\sum_{k=0}^{n}\binom{n}{k}h^{(k)}(\lambda)\mu^{n-k}\right). \]
    To je enako 0 natanko tedaj, ko je $\lambda$ $(n+1)$-kratna ničla funkcije $h$.
\end{proof}

\noindent Funkcija $h$ je analitična, zato zanjo veljajo naslednje lastnosti:
\begin{enumerate}
    \item[(1)] Za pojubno število $R > 0$ je množica ničel, ki zadoščajo pogoju $|\lambda|\leq R$ končna. Torej je
            množica ničel funkcije $h$ števna množica.
    \item[(2)] Naj bo množica ničel karakteristične enačbe neskončna. 
    Torej gre za zaporedje, ki ga lahko označimo z $\left\{\lambda_n\right\}$. Potem velja
    $|\lambda_{n}| \to \infty$. Ker je $|\beta|e^{-\text{Re}(\lambda_{n})}=|\lambda_n|$, sledi, da
    $\text{Re}(\lambda_n) \to -\infty$. Prav tako za vsak $a \in \R$ obstaja končno mnogo ničel $\lambda_n$,
    da je $\text{Re}(\lambda_n) \geq a$. 
    \item[(3)] Če je $\lambda$ ničla funkcije $h$, je njen red končen.
    \item[(4)] Če je $\lambda$ ničla funkcije $h$, je tudi $\overline{\lambda}$ ničla.
\end{enumerate}


\begin{trditev}
    Za karakteristično funkcijo $h$ velja naslednje:
    \begin{enumerate}
        \item Če je $\beta < 0$, potem obstaja natanko ena realna ničla $\lambda$ in je pozitivna.
        \item Če je $0 < \beta < 1/e$, potem obstajata natanko dve različni negativni realni
         ničli, $\lambda_{1} < \lambda_{2}$, za kateri velja:
         \[ \lim_{\beta \to 0}\lambda_{1}=-\infty \quad \text{in} \quad \lim_{\beta \to 0}\lambda_{2}=0. \]
        \item Če je $\beta = 1/e$, potem je edina realna ničla $\lambda = -1$ in je dvakratna.
        \item Če je $\beta > 1/e$, potem funkcija nima realnih ničel.
    \end{enumerate}
\end{trditev}

\begin{proof}[Dokaz]
    Trditev se v celoti osredotoča na primer, ko je $\lambda\in\R$ zato bo to tudi naš privzetek.
    Nanizajmo dokaze posameznih točk:
    \begin{enumerate}
        \item Funkcija $h$ je naraščajoča vzdolž realne osi in velja $h \to \pm \infty$, ko $\lambda \to \pm \infty$. Obstaja 
        torej natanko ena realna ničla. Ker je $\beta < 0$, bo za tako ničlo veljalo tudi $\lambda = -\beta e^{-\lambda} >0$.
        \item Funkcija $h$ ima pri $\lambda=\ln\beta$ stacionarno točko z negativno lastno vrednostjo. Po drugi strani pa velja, da je gre $h\to\infty$ za $\lambda\to\pm\infty$
        , zato obstajata natanko dve realni ničli. Ker je 
        $\beta > 0$, za eno izmed njiju velja $\lambda = -\beta e^{-\lambda} <0$. Drugi del trditve sledi iz dejstva, da velja:
        $ \beta = -\lambda e^{\lambda} \to 0$ natanko tedaj, ko gre $\lambda \to 0$ ali $\lambda \to -\infty$.
        \item Kot v prejšnjem primeru imamo tudi tokrat stacionarno točko, ki pa je hkrati enaka ničli. 
        Tedaj je taka ničla enaka $\lambda = -1$. Velja $h'(-1)=0$ in $h''(-1)=1$, torej
        je ničla reda 2.
        \item Stacionarna točka ima v tem primeru pozitivno vrednost. Ker je na intervalu pred njo funkcija padajoča,
         za njo pa padajoča, realne ničle ne obstajajo. 
    \end{enumerate}
\end{proof}

Za $\beta > 0$ ima enačba lepe lastnosti, kar nam pove naslednja trditev.

\begin{trditev}
    Za karakteristično funkcijo $h$ velja:
    \begin{enumerate}
        \item Če je $0<\beta < \pi /2$, potem obstaja $\delta > 0$, da je Re($\lambda$) $\leq -\delta$
        za vse ničle $\lambda$.
        \item Če je $\beta = \pi /2$, potem sta $\lambda_{1,2}=\pm i\pi /2$ ničli reda 1.
        \item Če je $\beta > \pi /2$, potem so ničle oblike $\lambda = a + ib$, kjer je $a>0$ in 
         $b \in (\pi /2, \pi)$.
    \end{enumerate} 
\end{trditev}

\begin{proof}[Dokaz]
(1)  Iz $\beta > 0$ sledi: Če obstaja $\lambda = a +ib$ z $a \geq 0$ in $b > 0$, ki reši sistem za $a$ in $b$,
je $\cos(b) \leq 0 < \sin(b)$. Torej mora biti $b \in S = \bigcup_{n=0}^{\infty}\{[\pi / 2,\pi)+2n\pi\}.$ Velja
tudi $\frac{\sin(b)}{b}=\frac{e^{a}}{\beta}$. Ker je tudi $\frac{d}{db}\frac{\sin(b)}{b} < 0$ za $b \in S$
in $\frac{\sin(\pi /2)}{\pi / 2} = \frac{2}{\pi}$, je torej $\frac{\sin(b)}{b} \leq \frac{2}{\pi}$
za vsako število $b \in S$.
Dobili smo naslednjo neenakost: 
\[\frac{1}{\beta} \leq \frac{e^a}{\beta} = \frac{\sin(b)}{b} \leq \frac{2}{\pi}.\]
Sledi, da je $\beta \geq \pi/2$. Torej, če je $\beta < \pi/2$, je potem Re($\lambda$) $<0$ za vsako ničlo
$\lambda$. Ko upoštevamo še lastnost (B) funckije $h$, je prvi del trditve dokazan. \\
(2)  Drugi del trditve se dokaže z direktnim izračunom. \\
(3)  Za dokaz tretjega dela nastavimo $\lambda = re^{i\theta}$ in vstavimo v karakteristično enačbo, pri čemer
upoštevamo Eulerjevo formulo:
\[ r(\cos(\theta - \pi)+i\sin(\theta - \pi))=\beta e^{-a}(\cos(-b)+i\sin(-b)).\]
Veljati mora torej:
\[r=\beta e^{-a} \quad \text{in} \quad \theta - \pi = -b + 2k\pi, \quad k \in \Z.\]
Rešitve enačbe iščemo v prvem kvadrantu, torej $a(\theta)>0,\quad b(\theta) = \pi - \theta > 0, \quad
k=0$ in $\theta \in (0,\pi/2)$. Ko upoštevamo še $y/x = \tan(\theta)$, dobimo naslednjo družino rešitev:
\begin{equation*}
    \begin{split}
        x(\theta) &= (\pi - \theta)\cot(\theta), \\
        y(\theta) &= \pi - \theta, \quad 0<\theta <\pi/2, \\
        \beta(\theta) &= \frac{\pi - \theta}{\sin(\theta)}e^{a(\theta)}. 
    \end{split}      
\end{equation*}
Tu so $a, b$, in $\beta$ zvezne funkcije spremenljivke $\theta$ na intervalu $(0,\pi/2)$. Ko gre $\theta \to 0$,
velja:
\[ a(\theta) \to \infty, \quad b(\theta) \to \pi, \quad \beta(\theta) \to \infty.\]
Za $\theta \to \pi/2$ pa je:
\[a(\theta) \to 0, \quad b(\theta) \to \pi/2, \quad \beta(\theta) \to \pi/2.\]
Ker je $\beta$ strogo padajoča na $(0,\pi/2)$, sledi da je njena zaloga vrednosti $(\pi/2, \infty)$.
\end{proof}

\begin{definicija}
    Dana je DDE $\dot{x}(t)=f(t,x(t-\tau))$ in njena rešitev $g(t)$.
    Pravimo, da je $g$ stabilna rešitev, če 
    za vsak $\varepsilon > 0$ obstaja $\delta > 0$ tako da za poljubno drugo rešitev $\psi$ z lastnostjo
    $|\psi(t_0)-g(t_0)| < \delta$, velja $|\psi(t)-g(t)| < \varepsilon$ za $t \geq t_0$.
    Če dodatno velja
    še $\lim_{x \to \infty}|\psi(t)-g(t)|=0$, pravimo, da je rešitev asimptotsko stabilna.
\end{definicija}


\begin{trditev}
    Za DDE oblike $\dot{x}(t)= -\alpha x(t-\tau), \tau >0, \alpha \in \R$, velja:
    \begin{enumerate}
        \item Če je $\alpha < 0$, je $x=0$ nestabilna rešitev.
        \item Če je $0<\alpha \tau <\pi/2$, je $x=0$ asimptotsko stabilna.
        \item Če je $\alpha \tau = \pi/2$, sta rešitvi $x=\sin(\pi(t/\tau)/2)$ in $x=\cos(\pi(t/\tau)/2)$.
        \item Če je $\alpha\tau > \pi/2$, je $x=0$ nestabilna.
    \end{enumerate}
\end{trditev}

\noindent \textit{Opomba: }Dokaz sledi direktno iz prejšnjih trditev in rezultata, ki pove, da je $x=0$ asimptotsko stabilna, če 
je $\text{Re}(\lambda) < 0$ za vse $\lambda$, ki so ničle karakteristične funkcije, oziroma nestabilna,
če obstaja ničla $\lambda$ s pozitivnim realnim delom.

\begin{trditev}
    Rešitve DDE oblike $\dot{x}(t)= -\alpha x(t-\tau), \tau >0, \alpha \in \R$,
    oscilirajo natanko tedaj, ko je $\alpha\tau>1/e.$
\end{trditev}

\begin{proof}
    Za vsak $\lambda \in \R$ je $U(\mu)=e^{\lambda\mu}$ bodisi monotona, bodisi konstantna funkcija.
    Za realne $\lambda$ torej rešitev $x(t)$ ne oscilira. Funkcija $x(t)$ je torej oscilirajoča natanko tedaj, ko
    je $\lambda\in\mathbb{C}\setminus\mathbb{R}$. Po trditvi 4.3 je to res natanko tedaj, ko je 
    $\beta = \alpha\tau > 1/e$.
\end{proof}

\section{Reševanje DDE z Laplaceovo transformacijo}



V tem poglavju si bomo ogledali, kako si lahko pri reševanju DDE pomagamo z Laplaceovo transformacijo.
Da bi razumeli postopek, najprej ponovimo in predstavimo nekaj lastnosti omenjene transformacije. Poglavje je 
povzeto po viru \cite{knjiga}.

\begin{definicija}
    Naj bo $f:[0,\infty) \to \R$ zvezna funkcija. Laplaceova transformacija funkcije $f$ je kompleksna funkcija
    $F$ podana s predpisom
    \[ F(s)=\int_{0}^{\infty}f(t)e^{-st}dt.\]
\end{definicija}

\noindent V zgornji definiciji funkcije $F$ nismo podali njenega definicijskega območja, zato se je
najprej potrebno vprašati, za katere $s\in\C$, je funkcija $F$ sploh definirana. Predpostavimo, da je
$|f(t)|\leq e^{at}$ za $a\in\R$ in $t\geq0$. Ker za $s\in\C$ velja $|e^{s}|=e^{\text{Re}(s)}$, imamo
\[|e^{-st}f(t)| \leq ce^{-t\text{Re}(s)}e^{at}=ce^{-t(\text{Re}(s)-a)}.\]
Če funkcijo na desni strani te neenačbe integriramo po $t\in [0,\infty)$, dobimo integral, ki je konvergenten za $\text{Re}(s)>a$.
Laplaceova transformacija je torej preslikava, ki funkciji, definirani na $[0,\infty)$, priredi
kompleksno funkcijo, definirano na polravnini $\{s\in\C | \text{Re}(s)>a\}$, za ustrezen $a$.

\begin{zgled}
    Izračunajmo Laplaceovo transformacijo funkcije $f(t)=t$:
    \[F(s)=\int_0^{\infty}te^{-st}dt = \frac{-ste^{-st}-e^{-st}}{s^2}\Bigr|_{0}^{\infty}=\frac{1}{s^2}.\]
    Dobljena funkcija je definirana za vse $s\in\C$ s pozitivnim realnim delom.
\end{zgled}


Izkazalo se bo, da je za uporabo Laplaceove transformacije pri reševanju DDE ključno, 
ali znamo poiskati tudi njen inverz.
Recimo, da imamo podano funkcijo
$F$ in želimo poiskati funkcijo $f$, katere Laplaceova transformacija je enaka $F$.
Zanimalo nas bo, kako to storiti in katerim pogojem mora zadoščati funkcija $f$.

Recimo, da je integral 
\[\int_0^{\infty}f(r)e^{-sr}dr\]
konvergenten za $\text{Re}(s)>\alpha$. Potem za $a>\alpha$ označimo
\[F(a+it)=\int_0^{\infty}f(r)e^{-(a+it)r}dr.\]
Obe strani pomnožimo z $e^{u(a+it)}$ in integriramo na $[-T,T]$ po spremenljivki $t$:
\[\int_{-T}^{T}e^{u(a+it)}F(a+it)dt=e^{au}\int_0^{\infty}f(r)e^{-ar}\left(\int_{-T}^{T}e^{iut-irt}dt\right)dr.\]
Zaradi absolutne konvergence smo lahko na desni strani zamenjali vrstni red integriranja.
Ko desno stran poenostavimo, dobimo:
\begin{equation} \label{eq2}
    \begin{split}
        \int_{-T}^{T}e^{u(a+it)}F(a+it)dt=2e^{au}\int_0^{\infty}f(r)e^{-ar}\frac{\sin(T(u-r))}{u-r}dr.
    \end{split}
\end{equation}
Funkciji $$k(u,r,T)=\frac{\sin(T(u-r))}{(u-r)}$$ pravimo Dirichlet-Kernelova funkcija in je velikega pomena 
pri iskanju inverzne funkcije. Da bo $k$ zvezna, dodajmo pogoj $k(u,u,T)=T$. % Izkaže se, da je za velike $T$ na
%območju $u=r$ vrednost integrala na desni vedno bolj odvisna le od izraza $f(r)e^{-ar}$. Tako dobimo
%idejo za iskanje funkcije $f$.

Izpeljimo sedaj formulo za reševanje enačbe (2). Integracijski interval razbijemo na tri 
dele: $$(0,\infty) = (0,u-d] \cup (u-d,u+d] \cup (u+d,\infty]$$ za nek $d>0$. Zaradi aditivnosti določenega
integrala dobimo:
\begin{equation*}
    \begin{split}
        \int_{-T}^{T}e^{u(a+it)}F(a+it)dt &= 2e^{au}\int_0^{u-d}g(u,r,T)dr +
        2e^{au}\int_{u-d}^{u+d}g(u,r,T)dr+ \\
        &+ 2e^{au}\int_{u+d}^{\infty}g(u,r,T)dr,
    \end{split}
\end{equation*}
kjer je $$g(u,r,T)=f(r)e^{-ar}\frac{\sin(T(u-r))}{(u-r)}.$$ Pokazati želimo, da se prvi in tretji integral
na desni strani enačbe bližata ničelni vrednosti, ko gre $T\to\infty$. Pri tem uporabimo naslednjo lemo.
\begin{lema}[Riemann-Lebesgueova lema]
    Če je funkcija $g$ absolutno integrabilna na realni osi, potem je:
    \[\lim_{T\to\infty}\int_{-\infty}^{\infty}g(t)\sin(Tt)dt=0.\]
\end{lema}
\noindent Dokaz leme lahko bralec najde v viru \cite{dokazleme}.

Ker smo predpostavili, da je $\int_0^{\infty}|f(t)|e^{-as}dt<\infty$, je torej funkcija 
$f(r)e^{-ar}(u-r)$ absolutno integrabilna na vsakem zaprtem intervalu, ki ne vsebuje točke $r=u$. Po 
Riemann-Lebesgueovi lemi se lahko znebimo prvega in tretjega integrala na desni strani enačbe. Ostane
nam torej le še:
\[I = 2e^{au}\int_{u-d}^{u+d}f(r)e^{-ar}\frac{\sin(T(u-r))}{u-r}dr.\]
Predpostavimo, da obstaja Taylorjev polinom druge stopnje za funkcijo $f$ na okolici točke $u$.
Zapišimo:
\[f(r)e^{-ar}=f(u)e^{-au}+h(u,r)(u-r),\]
kjer je $|h(u,r)|\leq k, \quad r\in(u-d,u+d).$
Vstavimo v $I$ in dobimo:
\[I = 2f(u)\int_{u-d}^{u+d}\frac{\sin(T(u-r))}{u-r}dr+2e^{au}\int_{u-d}^{u+d}h(u,r)\sin(T(u-r))dr.\]
Ker je $|\sin(T(u-r))|\leq1$, je drugi integral reda $\mathcal{O}(d)$, ko gre $T\to\infty$. Z uvedbo 
nove spremenljivke $v=T(u-r)$, dobimo
\[I=2f(u)\int_{-Td}^{Td}\frac{\sin v}{v}dv +\mathcal{O}(d),\]
oziroma
\[\lim_{T\to\infty}I=2f(u)\int_{-\infty}^{\infty}\frac{\sin v}{v}dv +\mathcal{O}(d).\]
Vrednost zgornjega integrala je enaka $\pi$, zato za vsak $d>0$ velja:
\[\lim_{T\to\infty}I=2\pi f(u)+\mathcal{O}(d).\]
Ker je $d$ poljuben, lahko trdimo, da velja:
\[\lim_{T\to\infty}\int_{-\infty}^{\infty}e^{(a+it)u}F(a+it)dt=2\pi f(u).\]
S tem smo predstavili glavne korake za dokaz izreka, ki je naveden spodaj, pri čemer pa 
smo predpostavili, da je $f$ dvakrat zvezno odvedljiva oz. da lahko poiščemo njen Taylorjev polinom drugega
reda v vsaki točki. 
Z nekaj dodatne analize lahko dokažemo tudi splošnejšo verzijo, ki je navedena spodaj. Dokaz najdemo v viru \cite{knjiga}.

\begin{izrek}
    Naj bo $f$ zvezna funkcija z naslednjimi lastnostmi:
    \begin{enumerate}
        \item[(1)]Integral $\int_0^{\infty}f(t)e^{-at}dt$ je absolutno konvergenten za nek $a>0$,
        \item[(2)]$f$ ima omejen odvod v okolici točke $u>0$.  
    \end{enumerate}
    Potem obstaja
    \[F(s)=\int_0^{\infty}e^{-st}f(t)dt\]
    za $\text{Re}(s)\geq a$. Za $b>a$ pa velja:
    \[\lim_{T\to\infty}\frac{1}{2\pi}\int_{-T}^{T}e^{(b+it)u}F(b+it)dt=f(u).\]
\end{izrek}


S tem smo dobili formulo za izračun inverzne funkcije pri Laplaceovi transformaciji.
Če vpeljemo novo spremenljivko $s=b+ir$ za $r\in[-T,T]$, lahko formulo še malo polepšamo:
\[\frac{1}{2\pi}\int_{-T}^{T}e^{(b+it)u}F(b+it)dr=\frac{1}{2\pi i}\int_{b-iT}^{b+iT}F(s)e^{st}ds.\]
Desni integral je krivuljni interal po daljici med točkama $b-iT$ in $b+iT$. Označimo
\[ \int_{(b)}F(s)e^{st}ds=\lim_{T\to\infty}\frac{1}{2\pi i}\int_{b-iT}^{b+iT}F(s)e^{st}ds,\]
kadar desna stran obstaja.
Končna formula za inverzno funkcijo je torej:
\[f(t)=\int_{(b)}F(s)e^{st}ds.\]


    Sedaj si na konkretnem zgledu poglejmo, kako si lahko z Laplaceovo transformacijo pomagamo pri reševanju DDE.
    \begin{equation} \label{eq3}
        \begin{split}
            \dot{x}(t)=x(t-1), \quad x(t)=\varphi(t), \quad t\in[-1,0].
        \end{split}
    \end{equation}
    Obe strani pomnožimo z $e^{-st}$ in integriramo na intervalu $[0,\infty)$:
     \[\int_{0}^{\infty}\dot{x}(t)e^{-st}dt = \int_{0}^{\infty}x(t-1)e^{-st}dt.\]
    Z uvedbo nove spremenljivke $u=t-1$ preoblikujemo desno stran enačbe:
    \begin{equation*}
        \begin{split}
            \int_{0}^{\infty}x(t-1)e^{-st}dt &= \int_{-1}^{\infty}x(u)e^{-s(u+1)}du \\
             &= \int_{-1}^{\infty}x(t)e^{-s(t+1)}dt \\
             &= e^{-s}\left(\int_{0}^{\infty}x(t)e^{-st}dt + \int_{-1}^{0}x(t)e^{-st}dt\right). 
        \end{split}      
    \end{equation*}
    Na levi strani enačbe uporabimo integracijo per partes. Pri tem predpostavljamo, da za 
    rešitev velja $x(t)e^{-st} \to 0$, ko gre $t \to\infty$.
    \[\int_{0}^{\infty}\dot{x}(t)e^{-st}dt=-x(0)+s\int_{0}^{\infty}x(t)e^{-st}dt.\]
    Sedaj upoštevajmo še, da za $t\in[-1,0]$ velja $x(t)=\varphi(t)$. Enačbo (3) smo pretvorili na naslednjo obliko:
    \[(s-1)\int_{0}^{\infty}x(t)e^{-st}dt=\varphi(0)+\int_{-1}^0\varphi(t)e^{-st}dt.\]
    Za $s\neq 1$ dobimo torej:
    \[\int_{0}^{\infty}x(t)e^{-st}dt=\frac{\varphi(0)+\int_{-1}^0\varphi(t)e^{-st}dt}{s-1}.\]
    Izraz na desni je zvezna funkcija spremenljivke $s$. Označimo jo z $G(s)$. Po izreku o inverzni
    transformaciji dobimo:
    \[x(t)=\int_{(c)}G(s)e^{st}ds.\]
    Za razliko od 3. poglavja, nam tu rešitve ni uspelo
    izraziti z elementarnimi funkcijami. Kljub temu bi lahko z dodatno analizo obravnavali njene lastnosti.

\section{Linearne diferencialne enačbe z zamikom}

Poglavje je povzeto po viru \cite{angdiploma}.

\begin{definicija}
    Linearna DDE je podana na sledeči način:
    \begin{equation} \label{eq4}
        \begin{split}
            \dot{x}(t)=ax(t)=bx(t-\tau)+f(t),\quad a,b\in\R, \quad \tau>0.
        \end{split}
    \end{equation}
    Če je $f(t)=0$, je enačba homogena, sicer je nehomogena. 
\end{definicija}

V tem poglavju bomo obravnavali homogene linearne DDE. Za njihovo reševanje uporabimo nastavek
$x(t)=e^{\lambda t}c$, $\lambda \in\C$, $c\in\R\setminus\{0\}$, in ga vstavimo v enačbo:
\begin{equation*}
        0 = \lambda e^{\lambda t}c - ae^{\lambda t}c -be^{\lambda(t-\tau)}c.   
\end{equation*}
Ker $e^{\lambda t}c \neq0$ za vse $t\in\R$, je karakteristična funkcija
\[h(\lambda)=\lambda-a-be^{-\lambda\tau}.\]
Iščemo ničle karakteristične funkcije $h$. Izraz lahko z uvedbo novih spremenljivk še dodatno 
poenostavimo. Če nastavimo $z:=\lambda\tau$, $\alpha:=a\tau$ in $\beta:=b\tau$, se karakteristična
enačba prevede na
\[h(z)=z-\alpha-\beta e^{-z}=0.\]
Reševanje te karakteristične enačbe je podobno kot pri modelu eksponentne rasti z zamikom.
\begin{trditev}
    Za homogene DDE oblike \eqref{eq4} in rešitev $x=0$ velja:
    \begin{itemize}
        \item Če je $a+b>0$, potem je $x=0$ stabilna.
        \item Če je $a+b<0$ in $b\geq a$, potem je $x=0$ asimptotsko stabilna.
        \item Če je $a+b<0$ in $b<a$, potem obstaja $\tau^{*}>0$, tako da je $x=0$ 
        asimptotsko stabilna za $0<\tau<\tau^{*}$ in nestabilna za $\tau>\tau^{*}$.
    \end{itemize}
\end{trditev}
\noindent Dokaz trditve se nahaja v viru \cite{envir}

\subsection{Obstoj in enoličnost rešitve}
Linearne DDE so splošnejše od tistih, ki smo jih obravnavali v poglavju o eksistenčnem izreku.
Res, podane so s predpisom
\[\dot{x}(t)=f(t,x(t),x(t-\tau)), \quad t\in[\xi,\xi+\tau],\]
kjer je $\tau>0,$ $\xi\geq 0$, funkcija $f$ pa je linearna. Natančneje, v zvezi ne nastopa zgolj člen $x(t-\tau)$ ampak tudi člen $x(t)$. To pomeni, da direktna aplikacija eksistenčnega izreka zanje ni možna. Vseeno lahko obstoj rešitve dokažemo z uporabo eksistenčnega izreka za NDE. Res, denimo da velja
$$x(t)=\phi(t), t\in J_{-}.$$
Potem za $t\in[\xi,\xi+\tau]$ velja 
\begin{equation*}
    \begin{split}
        \dot{x}(t)&=f(t,x(t),\phi(t-\tau)), \\
        x(\xi) &= \phi(\xi).
    \end{split}
\end{equation*}
To je NDE prvega reda, na kateri lahko ob dodatni predpostavki, da je $\phi$ zvezna uporabimo eksistenčni izrek za navadne 
diferencialne enačbe, saj je $f$ linearna. Torej dobimo rešitev, ki je enolična na $[\xi,\xi+\tau]$. Sedaj postopamo 
induktivno in rešitev razširimo na naslednji interval $[\xi+\tau,\xi+2\tau]$.
Podobno kot pri že videnem dokazu eksistenčnega izreka lahko torej rešitev razširimo na $[\xi,\infty)$.
Ker se podobna ideja pojavi tudi v naslednjem razdelku, je za primer linearne DDE ne bomo bolj natančno analizirali.


\section{Linearni sistemi DDE}

V tem razdelku si bomo ogledali tudi sistem več diferencialnih enačb z zamikom. Ker bi bila analiza splošnega
sistema te oblike prezahtevna, se bomo omejili na linearne sisteme, za 
katere imamo dobro razvito teorijo tudi v primeru navadnih diferencialnih enačb. Poglavje je povzeto po viru 
\cite{angdiploma}.

\begin{definicija}
    Sistem linearnih DDE s konstantnimi koeficienti je podan z 
    \begin{equation} \label{eq5}
        \begin{split}
            \dot{x}(t)=Ax(t)+Bx(t-\tau)+f(t)\quad \text{za}\quad t\geq0,
        \end{split}
    \end{equation} kjer so
    $A,B\in\R^{n\times n}$, $\tau>0$ in $f$ zvezna funkcija. Začetni pogoj je znova podan s predpisom
    \begin{equation} \label{eq6}
        \begin{split}
            x(t)=\phi(t)\quad \text{za}\quad t\in [-\tau,0],
        \end{split}
    \end{equation}
    le da je $\phi$ tokrat zvezna vektorska funkcija. Če je $f(t)\equiv0$, pravimo,
    da je sistem homogen.
\end{definicija}
Za analizo linearnih sistemov DDE bomo uporabili Laplaceovo transformacijo. Najprej jo uporabimo na levi strani sistema:
\begin{equation*}
    \begin{split}
        G_1(s)&= \int_0^{\infty}e^{-st}\dot{x}(t)dt \\
            &= (e^{-st}x(t))\Bigr|_{0}^{\infty} + s\int_0^{\infty}e^{-st}x(t)dt \\
            &= -\phi(0) + sX(s),
    \end{split}      
\end{equation*}
kjer je $X(s)$ Laplaceova transformacija iskane funkcije.
Ko transformiramo še desno stran, dobimo:
\begin{equation*}
    \begin{split}
        G_2(s)&= \int_0^{\infty}e^{-st}\left(Ax(t)+Bx(t-\tau)+f(t)\right)dt \\
            &= A\int_0^{\infty}e^{-st}x(t)dt+B\int_0^{\infty}e^{-st}x(t-\tau)dt+\int_0^{\infty}e^{-st}f(t)dt \\
            &= AX(s) + B\left(\int_0^{\tau}e^{-st}\phi(t-\tau)dt+\int_{\tau}^{\infty}e^{-st}x(t-\tau)dt\right) + F(s) \\
            &= AX(s) + B\left(\int_0^{\tau}e^{-st}\phi(t-\tau)dt+\int_{0}^{\infty}e^{-s(t+\tau)}x(t)dt\right) + F(s) \\
            &= AX(s) + B\left(\int_0^{\tau}e^{-st}\phi(t-\tau)dt+e^{-st}X(s)\right)+F(s) \\
            &= (A+e^{-st}B)X(s) + B\int_0^{\tau}e^{-st}\phi(t-\tau)dt+F(s) \\
            &= (A+e^{-st}B)X(s) + B\Phi(s)+F(s).
    \end{split}      
\end{equation*}
Pri izračunu smo predpostavili, da je funkcija $\phi$ ničelna na $(0,\infty)$. Funkcija $\Phi$ pa 
označuje Laplaceovo transformacijo funkcije $\phi(t-\tau)$.

Ko izenačimo $G_1$ in $G_2$, dobimo:
\[X(s)=K(s)(\phi(0)+B\Phi(s)+F(s)),\]
kjer je 
\[K(s)=(sI-A-e^{-st}B)^{-1}.\]
Da bi dobili lepo rešitev enačbe, moramo poiskati inverzno transformacijo funkcije $K$, ki jo bomo označili s $k$.
Spomnimo, da mora taka funkcija rešiti sistem \eqref{eq5} za $f=0$ in zadostiti začetnemu pogoju
\[
        \xi(\theta) =
        \begin{dcases}
            I &\quad \text{za  } \theta=0 \\
            0 &\quad \text{za  } \theta \in [-\tau,0). \\
        \end{dcases}
    \]
Kljub nezveznosti funkcije $\xi$ pri $\theta=0$ funkcija $k$ obstaja za $t\geq0$. Pravimo ji fundamentalna matrična
rešitev sistema \eqref{eq5}.

Naslednji izrek za bolj splošne sisteme DDE nam bo podal pomemben rezultat o enoličnosti in obstoju rešitve.

\begin{izrek}
    Podan je linearen sistem DDE s konstantnimi koeficienti:
    \begin{equation} \label{eq7}
        \dot{x}(t)=Ax(t)+Bx(t-\tau)=f(t)\quad \text{za}\quad t\geq\tau,
    \end{equation}
    kjer je $\tau>0$, pri začetnem pogoju
    \[x(t)=\phi(t), \quad t \in[0,\tau].\]
    Naj bo $\phi$ gladka. Potem obstaja natanko ena funkcija $x(t)$, ki 
    zadošča začetnemu pogoju in reši sistem (7) za $t\geq\tau$. Če je $f$ zvezna na $[0,\infty)$, je $x(t)$ 
    zvezno odvedljiva za vse $t>0$ razen morda v točkah oblike $t=n\tau$ za $n\in\N$.
\end{izrek}

\begin{proof}
    Definirajmo
    \[g(t)=f(t)-\dot{x}(t)+Ax(t)+Bx(t-\tau).\]
    Ker je $f$ zvezna na $[0,\infty)$ in $\phi$ gladka na $[0,\tau]$, je $g$ zvezna na $[0,2\tau]$. Po integraciji 
    sledi, da obstaja enolična funkcija $x(t)$, ki zadošča sistemu (7) na $(\tau,2\tau)$ in začetnemu pogoju. Ta 
    funkcija je zvezno odvedljiva na $[0,2\tau]$ razen morda v $t=\tau$ ali $t=2\tau$.
    Sledi, da je $g$ zvezna na $[\tau,3\tau]$. Z integracijo razširimo rešitev 
    na $[0,3\tau]$. Tu je rešitev zvezno odvedljiva razen morda za $t=n\tau$ za $n=1,2,3$. Postopek lahko 
    induktivno ponavljamo v neskončnost. Vidimo, da je rešitev zvezna na $[0,\infty)$ in zvezno odvedljiva razen 
    morda v točkah oblike $t=n\tau$ za $n\in\N$.
\end{proof}

\begin{lema}
    Rešitev sistema \eqref{eq5} pri začetnem pogoju \eqref{eq6} lahko izrazimo kot
    \[x(t):=x(t;\phi,f)=x(t;\phi,0)+x(t;0,f),\]
    kjer smo z $x(t;\phi,f)$ označili rešitev začetnega problema z nehomogenostjo $f$ in začetno funkcijo $\phi$.
\end{lema}

\begin{proof}
    Za $t\in [0,\tau]$ velja:
    \begin{equation*}
        \begin{split}
            \dot{x}(t;\phi,f) &= Ax(t;\phi,f)+B\phi(t-\tau)+f(t) \\
            \dot{x}(t;\phi,0)  &= Ax(t;\phi,0)+B\phi(t-\tau)\\
            \dot{x}(t;0,f)  &= Ax(t;0,f)+f(t).
        \end{split}      
    \end{equation*}
Za $t\in[0,\tau]$ je torej
\[\dot{x}(t;\phi,0)+\dot{x}(t;0,f)=A\left(x(t;\phi,0)+x(t;0,f)\right)+B\phi(t-\tau)+f(t).\]
Sledi, da je
\[x(t):=x(t;\phi,f)=x(t;\phi,0)+x(t;0,f).\]
Po enoličnosti rešitve je to res za $t\geq0$.
\end{proof}

Zaključimo z zgledom, ki ilustrira konstrukcijo rešitve iz eksistenčnega izreka za linearne DDE in linearne sisteme DDE.
\begin{zgled}
    Podan je homogen linearen sistem DDE
    \[\dot{x}=Ax(t)+Bx(t-\tau)\quad \text{za}\quad t\geq0,\]
    kjer je $A=
    \begin{bmatrix}
        0 & 1 \\
        -2 & 3 
    \end{bmatrix}$
    in $B=
    \begin{bmatrix}
        0 & 1 \\
        1 & 0 
    \end{bmatrix}.$
    Začetni pogoj je podan kot 
    $\phi(t)=\begin{bmatrix}
        1 \\
        1 
    \end{bmatrix}$
    za $t\in[-\tau,0]$. \\
    Najprej izračunamo lastne vrednosti matrike $A$ in pripadajoče lastne vektorje:
    \[\text{det}(A-\lambda I)=\lambda^2-3\lambda+2=0.\]
    Lastni vrednosti $A$ sta torej $\lambda_1=1$ in $\lambda_2=2$.
    Iz homogenih sistemov
    \[(A-\lambda_1 I)v_1=\begin{bmatrix}
        0 \\
        0 
    \end{bmatrix}
    \quad \text{in} \quad
    (A-\lambda_2 I)v_2=\begin{bmatrix}
        0 \\
        0 
    \end{bmatrix}\]
    dobimo lastna vektorja $v_1=\begin{bmatrix}
        1 \\
        1 
    \end{bmatrix}$
    in  $v_2=\begin{bmatrix}
        1 \\
        2 
    \end{bmatrix}$.
    Rešitve sistema 
    \[\dot{x}(t)=Ax(t)\]
    so torej 
    \[x_h(t)=c_1e^t\begin{bmatrix}
        1 \\
        1 
    \end{bmatrix}
    +c_2e^{2t}\begin{bmatrix}
        1 \\
        2 
    \end{bmatrix}\]
    za konstanti $c_1$ in $c_2$. 
    Za $t\in[0,\tau]$ je
    \begin{equation*}
        \begin{split}
            \dot{x}(t) &= Ax(t)+B\phi(t-\tau) \\
            &= \begin{bmatrix}
                0 & 1 \\
                -2 & 3 
            \end{bmatrix}x(t)+\begin{bmatrix}
                0 & 1 \\
                1 & 0 
            \end{bmatrix}
            \begin{bmatrix}
                1  \\
                1  
            \end{bmatrix} \\
            &= \begin{bmatrix}
                0 & 1 \\
                -2 & 3 
            \end{bmatrix}x(t)+
            \begin{bmatrix}
                1  \\
                1  
            \end{bmatrix}.
        \end{split}
    \end{equation*}
    Iščemo partikularno rešitev 
    \[x_p(t)=\begin{bmatrix}
        p_1  \\
        p_2  
    \end{bmatrix}\]
    za neki konstantni $p_1$ in $p_2$, ki reši ta sistem. $x_p(t)$ vstavimo v enačbo in dobimo
    \[\dot{x_p}(t)=\begin{bmatrix}
        0  \\
        0  
    \end{bmatrix} = \begin{bmatrix}
        0 & 1 \\
        -2 & 3 
    \end{bmatrix}
    \begin{bmatrix}
        p_1  \\
        p_2  
    \end{bmatrix} +
    \begin{bmatrix}
        1  \\
        1  
    \end{bmatrix} = 
    \begin{bmatrix}
        p_2+1  \\
        -p_1+3p_2+1  
    \end{bmatrix}.\]
    Sledi, da je $p_1=-2$ in $p_2=-1$. Partikularna rešitev je torej
    \[x_p(t)=
    \begin{bmatrix}
        -2  \\
        -1  
    \end{bmatrix}.\]
    Za $t\in[0,\tau]$ je splošna rešitev sistema
    \[x(t)=x_h(t)+x_p(t)= c_1e^t\begin{bmatrix}
        1 \\
        1 
    \end{bmatrix}
    +c_2e^{2t}\begin{bmatrix}
        1 \\
        2 
    \end{bmatrix} + \begin{bmatrix}
        -2  \\
        -1  
    \end{bmatrix}.\]
    Z upoštevanjem začetnega pogoja določimo še konstanti $c_1$ in $c_2$:
    \[x(0)=\begin{bmatrix}
        1  \\
        1  
    \end{bmatrix} = c_1 \begin{bmatrix}
        1  \\
        1  
    \end{bmatrix} + c_2 \begin{bmatrix}
        1  \\
        2  
    \end{bmatrix} + \begin{bmatrix}
        -2  \\
        -1  
    \end{bmatrix} = \begin{bmatrix}
        c_1 + c_2 -2  \\
        c_1+2c_2 -1  
    \end{bmatrix}.\]
    Dobimo, da je $c_1=4$ in $c_2=-1$. Končna rešitev je torej 
    \[x(t)=x_h(t)+x_p(t)= 4e^t\begin{bmatrix}
        1 \\
        1 
    \end{bmatrix}
    -e^{2t}\begin{bmatrix}
        1 \\
        2 
    \end{bmatrix} + \begin{bmatrix}
        -2  \\
        -1  
    \end{bmatrix}, \quad t\in[0,\tau].\]
    Poiščimo še rešitev za $t\in[\tau,2\tau]$. Rešitve sistema $\dot{x}=Ax(t)$ so še vedno oblike
    \[ x_h(t)=c_1e^t\begin{bmatrix}
        1 \\
        1 
    \end{bmatrix}
    +c_2e^{2t}\begin{bmatrix}
        1 \\
        2 
    \end{bmatrix}.\]
    Označimo
    \[\Psi(t)=\begin{bmatrix}
        4e^{t}-e^{2t}-2  \\
        4e^{t}-2e^{2t}-1  
    \end{bmatrix}.\]
    Za $t\in[\tau,2\tau]$ je
    \begin{equation*}
        \begin{split}
            \dot{x}(t) &= Ax(t)+Bx(t-\tau) \\
            &= Ax(t)+B\Psi(t-\tau) \\
            &= \begin{bmatrix}
                0 & 1 \\
                -2 & 3 
            \end{bmatrix}x(t)+\begin{bmatrix}
                0 & 1 \\
                1 & 0 
            \end{bmatrix}
            \begin{bmatrix}
                4e^{(t-\tau)}-e^{2(t-\tau)}-2  \\
                4e^{(t-\tau)}-2e^{2(t-\tau)}-1  
            \end{bmatrix} \\
            &= \begin{bmatrix}
                0 & 1 \\
                -2 & 3 
            \end{bmatrix}x(t)+
            \begin{bmatrix}
                4e^{(t-\tau)}-2e^{2(t-\tau)}-1  \\
                4e^{(t-\tau)}-e^{2(t-\tau)}-2  
            \end{bmatrix}.
        \end{split}
    \end{equation*}
    Ker bi bilo tu iskanje partikularne rešitve z nastavkom prezahtevno, uporabimo variacijo konstante.
    Matrika Wronskega je enaka
    \[W(t)=\begin{bmatrix}
        e^{\lambda_1t} & e^{\lambda_2t} \\
        (e^{\lambda_1t})' & (e^{\lambda_1t})' 
    \end{bmatrix}
    =\begin{bmatrix}
        e^t & e^{2t} \\
        e^t & 2e^{2t} 
    \end{bmatrix}.\]
    Partikularno rešitev iščemo v obliki $x_p(t)=W(t)D(t)$, $D(t)=(D_1(t),D_2(t)).$ Ko to vstavimo v enačbo, dobimo 
    \begin{equation*}
        \begin{split}
            \dot{W}D+W\dot{D} &= AWD+B\Psi(t) \\
            \dot{D}&=W^{-1}(B\Psi(t)) \\
            D &=\int W^{-1}(B\Psi(t))dt
        \end{split}
    \end{equation*}
    \overfullrule=0mm Za tak vektor $D$ bo torej $x_p(t)=W(t)D(t)$ partikularna rešitev sistema.
    Izračunajmo najprej $W^{-1}$:
    \[W^{-1}=\frac{1}{2e^{3t}-e^{3t}}\begin{bmatrix}
        2e^{2t} & -e^{2t} \\
        -e^t & e^t
    \end{bmatrix}=e^{-3t}\begin{bmatrix}
        2e^{2t} & -e^{2t} \\
        -e^t & e^t
    \end{bmatrix}.\] 
    Po formuli izračunamo še $D$:
    \begin{equation*}
        \begin{split}
            D &=\int W^{-1}(B\Psi(t))dt \\
                    &= \int W^{-1}(\begin{bmatrix}
                        4e^{t}-2e^{2t}-1  \\
                        4e^{t}-e^{2t}-2  
                    \end{bmatrix})dt \\
                    &= e^{-3t} \int \begin{bmatrix}
                        2e^{2t} & -e^{2t} \\
                        -e^t & e^t
                    \end{bmatrix} \begin{bmatrix}
                        4e^{t}-2e^{2t}-1  \\
                        4e^{t}-e^{2t}-2  
                    \end{bmatrix} dt \\
                    &= e^{-3t}\int \begin{bmatrix}
                        -3e^{4t}+4e^{3t}  \\
                        e^{3t}-e^t  
                    \end{bmatrix} dt \\
                    &= \int \begin{bmatrix}
                        -3e^{t}+4  \\
                        1-e^{-2t}  
                    \end{bmatrix} dt \\
                    &= \begin{bmatrix}
                        -3e^t+4t  \\
                        t+\frac{1}{2}e^{-2t}  
                    \end{bmatrix}.
        \end{split}
    \end{equation*}
    Partikularna rešitev je torej 
    \begin{equation*}
        \begin{split}
            x_p(t) &= WD \\
                    &= \begin{bmatrix}
                        e^t & e^{2t} \\
                        e^t & 2e^{2t} 
                    \end{bmatrix} \begin{bmatrix}
                        -3e^t+4t  \\
                        t+\frac{1}{2}e^{-2t}  
                    \end{bmatrix} \\
                    &= \begin{bmatrix}
                        te^{2t}-3e^{2t}+4te^t+\frac{1}{2}  \\
                        2te^{2t}-3e^{2t}+4te^{t}+1 
                    \end{bmatrix}.
        \end{split}
    \end{equation*}
    Za $t\in[\tau,2\tau]$ je splošna rešitev sistema torej
    \[x(t)=x_h(t)+x_p(t)=\begin{bmatrix}
        te^{2t}+(c_2-3)e^{2t}+4te^t+c_1e^t+\frac{1}{2}  \\
        2te^{2t}+(2c_2-3)e^{2t}+4te^{t}+c_1e^t+1 
    \end{bmatrix}.\]
    Za določitev konstant $c_1$ in $c_2$ uporabimo pogoj
    \[x(\tau) =\Psi(\tau)=\begin{bmatrix}
        4e^{\tau}-e^{2\tau}-2  \\
        4e^{\tau}-2e^{2\tau}-1  
    \end{bmatrix} = \begin{bmatrix}
        \tau e^{2\tau}+(c_2-3)e^{2\tau}+4\tau e^\tau+c_1e^\tau+\frac{1}{2}  \\
        2\tau e^{2\tau}+(2c_2-3)e^{2\tau}+4\tau e^{\tau}+c_1e^\tau+1 
    \end{bmatrix}. \]
    Rešimo sistem dveh enačb z dvema neznankama in dobimo 
    \begin{equation*}
        \begin{split}
            c_1 &= 4-e^{\tau}-3e^{-\tau}-4\tau e^{\tau}-4\tau \\
            c_2 &= \frac{1}{2}e^{-2\tau}-1-\tau.
        \end{split}
    \end{equation*}
    Končna rešitev sistema za $t\in[\tau,2\tau]$ je
    \[x(t)= \begin{bmatrix}
        te^{2t}+(\frac{1}{2}e^{-2\tau}-\tau-4)e^{2t}+4te^t+(4-e^{\tau}-3e^{-\tau}-4\tau e^{\tau}-4\tau)e^t+\frac{1}{2}  \\
        2te^{2t}+(e^{-2\tau}-2\tau-5)e^{2t}+4te^{t}+(4-e^{\tau}-3e^{-\tau}-4\tau e^{\tau}-4\tau)e^t+1 
    \end{bmatrix}.\]
    Z induktivnim postopkom bi dobili rešitev na poljubnem intervalu $[n\tau, (n+1)\tau],$ $n\in\N.$

\end{zgled}


Nazadnje izpeljimo še nekoliko splošnejši postopek za reševanje linearnih sistemov DDE oblike 
\[\dot{x}(t)=L(x_t),\]
kjer je $L:C\to\C^n$ za $C=\mathcal{C}([-\tau,0],\C^n)$ linearna preslikava, $x_t$ pa je definiran kot 
\[x_t(\theta):=x(t+\theta) \quad \text{za} \quad -\tau\leq\theta\leq0.\]
Rešitve sistema iščemo z nastavkom 
\[x(t)=e^{\lambda t}v, \quad v\neq0,\]
kjer je $\lambda\in\C$ in  $v\in\C^n$. Uvedimo še novo oznako $\text{exp}_{\lambda}(\theta):=e^{\lambda\theta}$.
Potem velja
\[x_t(\theta)=x(t+\theta)=e^{\lambda(t+\theta)}v=e^{\lambda t}e^{\lambda\theta}v=e^{\lambda t}\text{exp}_{\lambda}(\theta)v.\]
Ker je $L$ linearna preslikava in $e^{\lambda t}\in\C$ za vse $\lambda$ in $t$, dobimo z upoštevanjem nastavka naslednji
pogoj:
\[\dot{x}(t)=\lambda e^{\lambda t}v=L(x_t)=L(e^{\lambda t}(\text{exp}_{\lambda})v)=e^{\lambda t}L((\text{exp}_{\lambda})v).\]
Obe strani delimo z $e^{\lambda t}$ in dobimo 
\[ \lambda v=L((\okr)v).\]
Naj bo $\{e_1,e_2,\dots,e_n\}$ standardna baza prostora $\C^n$. Potem je 
$v=\sum_{i=1}^nv_ie_i$, kjer je $v_i\in\C$ za vse $i$. Iz linearnosti preslikave $L$ sledi 
\[L((\okr)v)=\sum_{i=1}^{n}v_i((\okr)e_i).\]
Definirajmo $n\times n$ matriko $L_\lambda$ na sledeči način:
\[ L_{\lambda}=[L((\okr)e_j)]_{j=1,\dots,n}.\]
$j$-ti stolpec matrike je torej vektor $L((\okr)e_j)$. Potem je
\begin{equation*}
    \begin{split}
        \lambda v &= L((\okr)v) \\
                &= \sum_{j=1}^{n}v_jL((\okr)e_j) \\
                &= (L((\okr)e_1)|(L((\okr)e_2)|\cdots|(L((\okr)e_n)) \cdot (v_1,v_2,\dots,v_n)^{T} \\
                &= L_{\lambda}v.
    \end{split}
\end{equation*}
Ker je torej $L_\lambda v =\lambda v$ in $v\neq0$, je $x(t)=e^{\lambda t}v$ rešitev sistema natanko tedaj, ko je
$\lambda$ lastna vrednost matrike $L_\lambda$, oziroma
\[ \text{det}(L_\lambda-\lambda I)=0,\]
vektor $v$ pa njen lastni vektor.
Poglejmo si znova homogen sistem 
\[ \dot{x}(t)=Ax(t)+Bx(t-\tau),\]
z $A$ in $B$ $n\times n$ matriki in zamik $\tau>0$. Linearno preslikavo $L$ definirajmo na naslednji način:
\[L(y)=Ay(0)+By(-\tau).\]
Matrika $L_\lambda$ je v tem primeru
\begin{equation*}
    \begin{split}
        L_\lambda &= (L_i((\okr)e_j)) \\
                &= ((A\okr(0)e_j+B\okr(-\tau)e_j)_i) \\
                &= ((Ae_j+e^{-\lambda\tau}Be_j)_i) \\
                &= (Ae_1+e^{-\lambda\tau}Be_1|Ae_2+e^{-\lambda\tau}Be_2|\cdots|Ae_n+e^{-\lambda\tau}Be_n) \\
                &= (A+e^{-\lambda\tau}B)(e_1,e_2,\dots,e_n) \\
                &= (A+e^{-\lambda\tau}B)I \\
                &= A+e^{-\lambda\tau}B.
    \end{split}
\end{equation*}
Karakteristična enačba sistema je torej
\[\text{det}(A+e^{-\lambda\tau}B-\lambda I)=0.\]
Rešitve sistema $\dot{x}(t)=Ax(t)+Bx(t-\tau)$ so torej oblike $x(t)=e^{\lambda t}v$, kjer je $\lambda$ lastna vrednost in $v$ pripadajoči lastni 
vektor matrike $A+e^{-\lambda\tau}B$. Opazimo, da je reševanje linearnih sistemov DDE precej bolj zahtevno kot
reševanje linearnih sistemov NDE, saj tu karakteristična funkcija ni več polinom.

\section{Logistična enačba z zamikom}
V tem poglavju bomo obravnavali logistično enačbo, ki opisuje rast prebivalstva ekosistema. Najprej bomo analizirali 
osnovno verzijo logistične enačbe, nato pa še enačbo z zamikom. Poglavje je povzeto po virih \cite{angdiploma}, 
\cite{logist} in \cite{slologist}.

Populacijo nekega ekosistema želimo opisati s funkcijo časa $x(t)$. Eksponentni model predpostavlja, da je rast 
populacije premo sorazmerna z njeno velikostjo, torej
\[\dot{x}(t)=ax(t), \]
kjer je $a$ konstanta. V tem primeru gre torej za eksponentno rast. Ker pa je ekosistem omejen z viri, se slej ko 
prej rast upočasni skoraj na 0, ko populacija doseže nosilnost okolja ($K>0$) za dano vrsto. Zato konstantno $a$ v 
enačbi zamenjamo z neko funkcijo $f(x)$, za katero velja $f(x)\approx a>0$ za majhne $x$, torej, da je rast na začetku 
približno eksponentna, ter $f(x)<0$ za velike $t$. Najpreprostejša funkcija s tema lastnostima je $f(x)=a-bx$, 
kjer je $b>0$. Vstavimo v enačbo in dobimo
\[\dot{x}(t)=(a-bx(t))x(t).\]
%Pogosto tu še izpostavimo $r$:
\begin{equation} \label{eq8}
    \dot{x}(t) = ax\left(1-\frac{x}{K}\right),
\end{equation}
kjer je $K=a/b$. Enačbi \eqref{eq8} pravimo logistična diferencialna enačba.

Sedaj izpeljimo še rešitve. Trivialni rešitvi sta $x(t)=0$ in $x(t)=K$. Da poiščemo še ostale, zapišemo enačbo v obliki
\[\frac{dx}{(1-\frac{x}{K})x}=a\text{ }dt.\]
Vidimo, da gre za enačbo z ločljivimi spremenljivkami. Z razcepom na parcialne ulomke dobimo
\[\left(\frac{1}{x}+\frac{\frac{1}{K}}{1-\frac{x}{K}}\right)dy=a\text{ }dt.\]
Tako smo enačbo poenostavili in lahko integriramo obe strani:
\[\ln|x|-\ln\biggr{|}1-\frac{x}{K}\biggr{|}=at+c.\]
Če je $0<x(t)<K$ za vse $t\geq0$, lahko absolutne vrednosti odstranimo. Dobimo 
\[\frac{x}{1-\frac{x}{K}}=e^ce^{at}.\]
Ko upoštevamo še začetni pogoj $x(0)=x_0$, dobimo končno rešitev
\[x(t)=\frac{x_0K}{x_0+(K-x_0)e^{-at}},\]
ki ji pravimo logistična funkcija.


\begin{figure}[h]
    \includegraphics[width=10cm, height=6cm]{1200px-Logistic-curve.svg.png}
    \caption{Logistična funkcija za $K=1$, $a=1$ in $x_0=0.5$}
\end{figure}

\noindent Za začetni pogoj $x(0)>K$ se izkaže, da je rešitev ista.
Opazimo, da je rešitev ustrezna tudi za $x_0=0$ in $x_0=K$. Iz 
\[\lim_{t\to\infty}x(t)=\frac{x_0K}{x_0}=K\]
pa vidimo, da se za $x_0\neq0$ velikost populacije z večanjem časa res bliža svoji nosilnosti. Torej je $x=K$ asimpotsko stabilna 
rešitev logistične enačbe, $x=0$ pa nestabilna, saj se poljubna rešitev, ki se začne blizu 0, v neskončnosti bliža $K$.


Navadna logistična enačba predpostavlja, da sprememba v velikosti populacije takoj vpliva na njeno rast. Da bi naredili 
model bolj realističen, je smiselno vpeljati logistično enačbo z zamikom. Ta je podana kot 
\[\dot{x}(t)=ax(t)\left(1-\frac{x(t-\tau)}{K}\right),\]
pri začetnem pogoju 
\[x(t)=x_0\quad\text{za }t\in[-\tau,0] \text{ in }x_0\in(0,K).\]
Zamaknjen je torej del, ki se nanaša na omejenost virov, prav tako pa predpostavljamo, da je začetno stanje na intervalu
med ničelnim in ravnovesnim stanjem. Opazimo, da za take $x(t)\in(0,K)$ in $x(t-\tau)\in(0,K)$ velja, da je
$\dot{x}(t)>0$, kar pomeni, da funkcija $x$ narašča do trenutka ko $x(t-\tau)$ preseže vrednost $K$. Za razliko od 
običajne logistične enačbe se to ob zadostnem zamiku dejansko zgodi(tudi v primeru začetne vrednosti, ki je manjša od $K$),
kar bomo potrdili v nadaljevanju.

Reševanja enačbe se lahko lotimo na že znani induktivni način. Za $t\in[0,\tau]$ velja 
\[\dot{x}(t)=ax(t)(1-bx_0) \Rightarrow x(t)=Ce^{a(1-x_0/K)t}.\]
Ker je $x(0)=x_0$, je torej 
\[x(t)=x_0e^{a(1-x_0/K)t}, \quad t\in[0,\tau].\]
Sedaj nadaljujemo za $t\in[\tau,2\tau]$:
\[\dot{x}(t)=ax(t)\left(1-\frac{x_0e^{a(1-x_0/K)(t-\tau)}}{K}\right).\]
Znova smo dobili enačbo z ločljivimi spremenljivkami, katere splošna rešitev je 
\begin{equation*}
    \begin{split}
        x(t) &= De^{a\int(1-\frac{x_0e^{a(1-x_0/K)t}}{K})dt} \\
             &= De^{at-\frac{x_0/K}{1-x_0/K}e^{a(1-x_0/K)t}}
    \end{split}
\end{equation*}
Z upoštevanjem pogoja $x(\tau)=x_0e^{a(1-x_0/K)\tau}$ lahko določimo tudi konstanto $D$ in nadaljujemo na naslednji 
interval. Že po dveh korakih je jasno, da kljub obstoju eksaktne rešitve ne bomo znali analizirati obnašanja funkcije 
$x(t)$ za velike čase, zato uporabimo drugačen način.

Pomagamo si z linearizacijo. To pomeni, da rešitev zapišemo v obliki $$x(t)=K+p(t),$$ pri čemer predpostavimo, da je 
funkcija $p(t)$ tako majhna, da lahko njene kvadratne člene zanemarimo. Spomnimo, zgoraj smo ugotovili:
\begin{itemize}
    \item $\dot{x}(t)>0$ za $x(t)>0$ in $x(t-\tau)<K$.
    \item $\dot{x}(t)<0$ za $x(t)>0$ in $x(t-\tau)>K$.
\end{itemize}
Pričakujemo torej, da bo vrednost $x=K$ stabilna ravnovesna točka, natančnejše obnašanje funkcije $x(t)$ okoli nje pa bo 
podala $p(t)$. Argumente za ustreznost tovrstne linearne analize ravnovesja lahko bralec najde v viru \cite{analiza}.
Ker lahko zanemarimo kvadratne člene funkcije $p(t)$, velja
\begin{equation*}
    \begin{split}
        (p(t)+K)'= p'(t) &=  a(K+p(t))\left(1-1-\frac{p(t-\tau)}{K}\right) \\
        p'(t) &= -ap(t-\tau)-\frac{a}{K}p(t)p(t-\tau) \\
        p'(t) &\approx -ap(t-\tau).
    \end{split}
\end{equation*}
Tako enačbo smo obravnavali v četrtem poglavju. Po trditvi 4.6 vemo, da $p(t)$ oscilira natanko tedaj, ko je 
$a\tau>1/e$.
Podobne zaključke lahko naredimo tudi za rešitve zamaknjene logistične enačbe v okolici ravnovesne točke $x=K$. 
Temu pritrjuje tudi skica na naslednji strani, ki je bila izdelana v okolju Matlab.
Numerična rešitev zamaknjene logistične enačbe je bila izračunana z osnovno enokoračno
difere\-nčno metodo s korakom
$h=0.01$ pri začetni vrednosti $x0=0.01$ in nosilnosti $K=1$.


\lstset{language=Matlab,%
    %basicstyle=\color{red},
    breaklines=true,%
    morekeywords={matlab2tikz},
    keywordstyle=\color{blue},%
    morekeywords=[2]{1}, keywordstyle=[2]{\color{black}},
    identifierstyle=\color{black},%
    stringstyle=\color{mylilas},
    commentstyle=\color{mygreen},%
    showstringspaces=false,%without this there will be a symbol in the places where there is a space
    numbers=left,%
    numberstyle={\tiny \color{black}},% size of the numbers
    numbersep=9pt, % this defines how far the numbers are from the text
    emph=[1]{for,end,break},emphstyle=[1]\color{red}, %some words to emphasise
    %emph=[2]{word1,word2}, emphstyle=[2]{style},    
}


\lstinputlisting{logist.m}

\begin{figure}[H]
    \includegraphics[width=\textwidth]{slikca6.eps}
\end{figure}

Vidimo, da se v vseh primerih, razen v zadnjem, vrednost $x(t)$ bliža nosilnosti $K=1$. Za $\alpha=\tau=1.5$, je namreč
$\alpha\tau=2.25>\pi/2$, zaradi česar je po trditvi 4.5 $x=K$ v tem primeru nestabilna rešitev. V ostalih primerih 
rešitev res niha okoli nosilnosti, kot smo pričakovali.



\overfullrule=0mm\printbibliography
\end{document}
