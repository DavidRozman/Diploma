\documentclass[12pt,a4paper]{amsart}
% ukazi za delo s slovenscino -- izberi kodiranje, ki ti ustreza
\usepackage[slovene]{babel}
%\usepackage[cp1250]{inputenc}
%\usepackage[T1]{fontenc}
\usepackage[utf8]{inputenc}
\usepackage{amsmath,amssymb,amsfonts,mathtools}
\usepackage{url}
%\usepackage[normalem]{ulem}
\usepackage[dvipsnames,usenames]{color}
\usepackage{biblatex}
\addbibresource{viri.bib}
\nocite{*}

% ne spreminjaj podatkov, ki vplivajo na obliko strani
\textwidth 15cm
\textheight 24cm
\oddsidemargin.5cm
\evensidemargin.5cm
\topmargin-5mm
\addtolength{\footskip}{10pt}
\pagestyle{plain}
\overfullrule=15pt % oznaci predlogo vrstico


% ukazi za matematicna okolja
\theoremstyle{definition} % tekst napisan pokoncno
\newtheorem{definicija}{Definicija}[section]
\newtheorem{primer}[definicija]{Primer}
\newtheorem{opomba}[definicija]{Opomba}
\newtheorem{zgled}[definicija]{Zgled}

\renewcommand\endprimer{\hfill$\diamondsuit$}


\theoremstyle{plain} % tekst napisan posevno
\newtheorem{lema}[definicija]{Lema}
\newtheorem{izrek}[definicija]{Izrek}
\newtheorem{trditev}[definicija]{Trditev}
\newtheorem{posledica}[definicija]{Posledica}


% za stevilske mnozice uporabi naslednje simbole
\newcommand{\R}{\mathbb R}
\newcommand{\N}{\mathbb N}
\newcommand{\Z}{\mathbb Z}
\newcommand{\C}{\mathbb C}
\newcommand{\Q}{\mathbb Q}
\newcommand{\okr}{\text{exp}_{\lambda}}


% ukaz za slovarsko geslo
\newlength{\odstavek}
\setlength{\odstavek}{\parindent}
\newcommand{\geslo}[2]{\noindent\textbf{#1}\hspace*{3mm}\hangindent=\parindent\hangafter=1 #2}

% naslednje ukaze ustrezno popravi
\newcommand{\program}{Finančna matematika} % ime studijskega programa: Matematika/Finan"cna matematika
\newcommand{\imeavtorja}{David Rozman} % ime avtorja
\newcommand{\imementorja}{doc.~dr. Uroš Kuzman} % akademski naziv in ime mentorja
\newcommand{\naslovdela}{Diferencialne enačbe z zamikom}
\newcommand{\letnica}{2022} %letnica diplome


% vstavi svoje definicije ...




\begin{document}

% od tod do povzetka ne spreminjaj nicesar
\thispagestyle{empty}
\noindent{\large
UNIVERZA V LJUBLJANI\\[1mm]
FAKULTETA ZA MATEMATIKO IN FIZIKO\\[5mm]
\program\ -- 1.~stopnja}
\vfill

\begin{center}{\large
\imeavtorja\\[2mm]
{\bf \naslovdela}\\[10mm]
Delo diplomskega seminarja\\[1cm]
Mentor: \imementorja}
\end{center}
\vfill

\noindent{\large
Ljubljana, \letnica}
\pagebreak

\thispagestyle{empty}
\tableofcontents
\pagebreak

\thispagestyle{empty}
\begin{center}
{\bf \naslovdela}\\[3mm]
{\sc Povzetek}
\end{center}
% tekst povzetka v slovenscini
V povzetku na kratko opi"site vsebinske rezultate dela. Sem ne sodi razlaga organizacije dela -- v katerem poglavju/razdelku je kaj, pa"c pa le opis vsebine.
\vfill
\begin{center}
{\bf Angle"ski naslov dela}\\[3mm] % prevod slovenskega naslova dela 
{\sc Abstract}
\end{center}
% tekst povzetka v anglescini
Prevod zgornjega povzetka v angle"s"cino.

\vfill\noindent
{\bf Math. Subj. Class. (2010):} navedite vsaj eno klasifikacijsko oznako -- dostopne so na \url{www.ams.org/mathscinet/msc/msc2010.html}  \\[1mm]  
{\bf Klju"cne besede:} navedite nekaj klju"cnih pojmov, ki nastopajo v delu  \\[1mm]  
{\bf Keywords:} angle"ski prevod klju"cnih besed
\pagebreak



% tu se zacne tekst seminarja
\section{Uvod}

Diferencialne enačbe spadajo med najpomembnejša področja matematične analize. So enačbe funkcije ene ali več
spremenljivk, ki povezujejo njene vrednosti z njenimi odvodi. Poleg matematike so ključne pri modeliranju tudi v drugih
znanstvenih vedah, npr. v fiziki, kemiji, astronomiji in ekonomiji.

Začetki diferencialnih enačb segajo v drugo polovico 17. stoletja, ko sta Isaac Newton in Gottfried Wilhelm von 
Leibniz predstavila infinitizemalni račun. V naslednjih sto letih so veliko dela preučevanju diferencialnih
enačb namenili največji svetovni matematiki, kot so Euler, Bernoulli in Lagrange, ter odkrili nekaj najpomembnejših
metod za reševanje.

Kljub mnogim raziskavam pa diferencialne enačbe še danes ostajajo kompleksen matematični problem. Podobno kot pri
nedoločenem integralu, splošni postopek za njihovo reševanje ne obstaja. Ker znamo reševati le določene tipe in je
eksplicitno iskanje rešitve v praksi redko izvedljivo, za reševanje pogosto uporabimo numerične metode,
včasih pa nas zanima le obstoj rešitve.

Navadne diferencialne enačbe povezujejo odvod funkcije v neki točki z njenimi trenutnimi vrednostmi.
V tej nalogi bomo obravnavali diferencialne enačbe z zamikom, pri katerih je vrednost odvoda iskane funkcije odvisna
tudi od njenih preteklih vrednosti. Motivacija za njihov študij je bila želja po bolj realističnih modelih za procese, 
katerih trenutno stanje je odvisno tudi od njihovih stanj v preteklosti. Take enačbe se zato pogosto uporabljajo
v biologiji, kemiji in mehaniki.

Na začetku naloge bomo na kratko ponovili osnove navadnih diferencialnih enačb. Nato bomo definirali
diferencialne enačbe z zamikom in si na preprostem zgledu pogledali primer reševanja in rešitev primerjali z 
rešitvijo analogne navadno diferencialne enačbe. V podrobnejši analizi bomo navedli eksistenčne izreke, torej kdaj 
rešitev obstaja ter kakšne lastnosti ima iskana funkcija. Za tem si bomo pogledali nekaj osnovnih tipov enačb z 
njihovimi postopki za reševanje. Na koncu pa bomo predstavili še nekaj zgledov uporabe takih enačb pri modeliranju.


\newpage

\section{Navadne diferencialne enačbe}

Za začetek se spomnimo osnov navadnih diferencialnih enačb.

\begin{definicija}
    Splošna diferencialna enačba reda $n$ je enačba oblike:
    \[F(x,y,y',y'',\dots,y^{(n)})=0,\]
    kjer je $F$ dana funkcija $n+2$ spremenljivk, $y$ neznana funkcija spremenljivke $x$, $y^{(k)}$ pa njeni
    odvodi.
\end{definicija}
Red enačbe je torej red najvišjega odvoda, ki nastopa v enačbi. Ker je $y$ funkcija ene spremenljivke,
njene vrednosti pa gledamo le v trenutni točki, pravimo taki enačbi \emph{navadna diferencialna enačba} (v
nadaljevanju NDE).

Poglejmo si primer zelo preproste NDE z ločljivimi spremenljivkami.

\begin{zgled}
    Podana je NDE z začetnim pogojem:
    \[\dot{x}(t) =-x(t),\quad x(0)=1.\]
    Rešimo jo s standardnim postopkom za enačbe z ločljivimi spremenljivkami:
    \begin{equation*}
    \begin{split}
      \frac{dx}{dt} & =-x \\
      \frac{dx}{-x} & =dt \\
      \int \frac{dx}{-x} & =\int dt \\
     -ln|x| & =t+c \\
      x(t) & =De^{-t} \\
    \end{split}
    \end{equation*}
Ko upoštevamo še začetni pogoj, dobimo $D=1$. Končna rešitev je torej \[x(t)=e^{-t}.\]
Rešitev je funkcija, ki je zvezna in odvedljiva na realni osi.
\end{zgled}

Pri NDE opazimo dve pomembni lastnosti, ki v nadaljevanju pri diferencialnih enačbah z zamikom ne bosta veljali. 
Vrednost odvoda funkcije $x$ v točki $t$ je bila odvisna le od njene vrednosti v tej isti točki. Poleg tega smo 
imeli še en začetni pogoj, zaradi katerega smo dobili enolično rešitev. Ta pogoj je bil podan kot vrednost funkcije
$x$ v eni točki. Običajna verzija eksistenčnega izreka pove, da je rešitev tovrstnega začetnega problema
ena sama, ker desna stran izpolnjuje Lipschitzev pogoj. Splošneje pa vemo, da
je splošna rešitev NDE reda $n$, če obstaja, $n$-parametrična.
Z vsakim dodatnim začetnim pogojem, ki je podan kot vrednost funkcije ali enega izmed njenih odvodov v fiksni točki,
se število parametrov v splošni rešitvi zmanjša za ena.

\newpage

\section{Diferencialne enačbe z zamikom}

\begin{definicija}
    Naj bo $\tau > 0$ konstanta in $J = [\xi,\xi +a]$, kjer je $\xi \geq 0$ in $a > 0$. Enačbi oblike
    \[\dot{x}(t)=f(t,x(t-\tau))\, \quad \text{za } t \in J\]
    pravimo diferencialna enačba prvega reda z zamikom (V nadaljevanju jo bomo 
    označevali s kratico DDE, ki ustreza angleškemu prevodu njenega imena - 
    delay differential equation.), kjer je $\tau$ zamik.
    Začetni pogoj je podan kot
    \[x(t)=\phi(t) \quad \text{za } t \in J_{-} = [\xi-\tau,\xi],\]kjer je $\phi$ zvezna funkcija.
\end{definicija}

\noindent\textit{Opomba:} V nadaljevanju diplomske naloge, bo izraz začetni problem pomenil diferencialno enačbo z zamikom, podano kot v 
zgornji definiciji.

Pomembna razlika v primerjavi z NDE, ki jo takoj opazimo, je začetni pogoj, ki v tem primeru ni več vrednost 
funkcije v eni točki, temveč kar funkcija sama na prejšnjem intervalu. Zaradi tega lahko sklepamo, da je družina vseh rešitev DDE neskončno
dimenzionalna.

\begin{zgled}
    Rešimo sledečo DDE:
    \[\dot{x}(t)=-x(t-\tau), \quad \text{za } t \geq 0,\]
    pri začetnem pogoju \[x(t)=1, \quad t \in [-\tau,0].\]
    Najprej rešimo za $t \in [0,\tau]$. Tedaj je $t-\tau \in [-\tau,0]$, zato je
    \[\dot{x}(t)=-x(t-\tau)=-1.\]
    Iz tega sledi
    \[x(t)=x(0)+ \int_{0}^{t}(-1)ds = 1-t, \quad \text{za } t \in [0,\tau].\]
    Tako smo dobili rešitev za interval $[0,\tau]$
    Analogno nadaljujemo za $t \in [\tau, 2\tau]$. Tu velja $t-\tau \in [0,\tau]$, torej
    \[\dot{x}(t)=-x(t-\tau)=-(1-(t-\tau))=(t-\tau)-1.\]
    Od tod sledi
    \[x(t)=x(\tau)+\int_{\tau}^{t}((s-\tau)-1)ds=1-t+\frac{(t-\tau)^2}{2},\quad \text{za }t\in[\tau,2\tau].\]
    Dokažimo sedaj, da za vsak interval $[(n-1)\tau,n\tau], n\in \N$ velja
    \[x(t)=1+\sum_{k=1}^{n}(-1)^k\frac{(t-(k-1)\tau)^k}{k!}.\]
    Za $n=1$ dobimo: $x(t)=1-t$, kar je v skladu z že izračunano rešitvijo.
    Predpostavimo sedaj, da formula drži za nek $n \in \N$ in dokažimo, da potem velja tudi za $n+1$.
    Ker je $t \in [n\tau,(n+1)\tau]$, je $t-\tau \in [(n-1)\tau, n\tau]$. Zato je:
    \[ \dot{x}(t)=-x(t-\tau)=-\left(1+\sum_{k=1}^{n}(-1)^k\frac{(t-\tau-(k-1)\tau)^k}{k!}\right).\]
    Iz tega sledi:
    \begin{equation*}
    \begin{split}
        x(t) &= x(n\tau) - \int_{n\tau}^{t}\left(1+\sum_{k=1}^{n}(-1)^k\frac{(s-k\tau)^k}{k!}\right)ds \\
        & = x(n\tau) - \int_{n\tau}^{t}ds - \int_{n\tau}^{t}\sum_{k=1}^{n}(-1)^k\frac{(s-k\tau)^k}{k!}ds \\
        & = x(n\tau) - t + n\tau - \sum_{k=1}^{n}\frac{(-1)^k}{k!}\int_{n\tau}^{t}(s-k\tau)^{k}ds \\
        & = x(n\tau) - t + n\tau - \sum_{k=1}^{n}\frac{(-1)^k}{k!}\left(\frac{(s-k\tau)^{k+1}}{k+1}\Bigr|_{n\tau}^{t}\right) \\
        & = x(n\tau) - t + n\tau - \sum_{k=1}^{n}\frac{(-1)^k}{(k+1)!}\left((t-k\tau)^{k+1}-((n-k)\tau)^{k+1}\right) \\
        & = 1 + \sum_{k=1}^{n}\left((-1)^k\frac{(n\tau-(k-1)\tau)^k}{k!}\right)- t + n\tau - \sum_{k=1}^{n}\frac{(-1^k)}{(k+1)!}
        (t-k\tau)^{k+1} - \\
        &- \sum_{k=1}^{n}\frac{(-1)^{k+1}}{(k+1)!}((n-k)\tau)^{k+1}.
    \end{split}      
    \end{equation*}
    Ko združimo prvo in tretjo vsoto dobimo $-n\tau$. Druga vsota pa skupaj z $-t$ da
     $\sum_{k=1}^{n+1}(-1)^{k}\frac{(t-(k-1)\tau)^k}{k!}$. Končno dobimo:
     \[x(t) = 1+ \sum_{k=1}^{n+1}(-1)^{k}\frac{(t-(k-1)\tau)^k}{k!},\]
    kar je pa ravno predpostavljena formula za interval $[n\tau, (n+1)\tau].$
    Dobljena rešitev je enolična, zvezna na $[-\tau,\infty)$, ni pa odvedljiva v točki $t=0$, saj je 
    levi odvod enak 0, desni pa -1. To nakazuje, da gladka odvisnost odvoda $\dot{x}(t)$ od zamaknjene
    funkcije $x(t-\tau)$
    ne zagotavlja gladkosti rešitve, ki bi jo dobili
    v NDE za $\tau=0$. Še podrobneje se bomo s tovrstno analizo ukvarjali v naslednjem razdelku.
\end{zgled}



\section{Obstoj in enoličnost rešitve}

\begin{izrek}
    Dan je začetni problem, kjer je $f$ zvezna funkcija na traku $S=J \times \R$, $\phi$ zvezna na $J_{-}$ in $\tau >0$
    konstanta. Potem obstaja natanko ena rešitev začetnega problema.
\end{izrek}

\begin{proof}[Dokaz]
    Naj bo \[\dot{x_{n}}(t)=f(t,x_{n}(t-\tau)) \text{ za } t \in [\xi + (n-1)\tau,\xi + n\tau],\]
    kjer je $n \in \N$.
    Iz $t \in [\xi, \xi + \tau]$ sledi $t-\tau \in [\xi - \tau, \xi]$. Za rešitev $x_{1}$ mora torej veljati:
    \[\dot{x_{1}}=f(t,x_{1}(t-\tau))=f(t,\phi(t-\tau)),\quad x_{1}(\xi)=\phi(\xi).\]
    Iz tega sledi, da je
    \begin{equation*}
        \begin{split}
            x_{1}(t) &= x_{1}(\xi) + \int_{\xi}^{t}f(s, x_{1}(s-\tau))ds \\
            & = \phi(\xi) + \int_{\xi}^{t}f(s,\phi(s-\tau))ds.
        \end{split}      
        \end{equation*}
    Funkcija $x_{1}$ je torej enolično definirana na intervalu $[\xi, \xi + \tau]$. Analogno lahko pokažemo, da velja
    \[\dot{x_{2}}=f(t,x_{2}(t-\tau))=f(t,x_{1}(t-\tau)) \text{ za } t \in [\xi + \tau, \xi + 2\tau]\] in 
    \[x_{2}(\xi + \tau)=x_{1}(\xi + \tau).\]
    Torej velja:
    \begin{equation*}
        \begin{split}
            x_{2}(t) &= x_{2}(\xi + \tau) + \int_{\xi + \tau}^{t}f(s, x_{2}(s-\tau))ds \\
            & = x_{1}(\xi + \tau) + \int_{\xi + \tau}^{t}f(s, x_{1}(s-\tau))ds.
        \end{split}      
        \end{equation*}
    Funkcija $x_{2}$ je torej enolično določena na intervalu $[\xi + \tau,\xi + 2\tau]$. Induktivno sklepamo, da lahko za vsak 
    $n \in \N$ definiramo $x_{n}$ s pomočjo $x_{n-1}$. Torej je $x_{n}$ enolično določena na intervalu 
    $[\xi + (n-1)\tau, \xi + n\tau]$ za vsak $n \in \N$. Povedano drugače, funkcija podana s predpisom
    \[
        x(t) =
        \begin{dcases}
            \phi(t) &\quad \text{za  } t \in J_{-}=[\xi - \tau, \xi] \\
            \phi(\xi) + \int_{\xi}^{t}f(s,x(s-t))ds &\quad \text{za  } t \in J=[\xi, \xi +a] \\
        \end{dcases}
    \]
    je dobro definirana za vse $t \in J_{-} \cup J = [\xi - \tau, \xi + a]$ in je rešitev začetnega problema. Iz 
    postopka pa sledi, da je enolična.
\end{proof}

\section{Model eksponentne rasti z zamikom}

V tem poglavju si bomo pogledali nekoliko splošnejšo verzijo DDE iz 3. poglavja, ki je podana na sledeči način:
\[\dot{x}(t)= \alpha x(t-\tau), \quad \tau > 0,\quad \alpha \in \R.\]
Enačba je odvisna od dveh parametrov, zato jo zaradi lažjega reševanja zreduciramo na enega.
Z uvedbo nove spremenljivke $\mu:=\eta t, \eta > 0$, in funkcijo
$U(\mu)=x(t)$ želimo enačbo prevesti na tako, kjer bo zamik $\tau$ enak 1:
\[\frac{dU}{d\mu}=\frac{dx}{\eta dt}=-\alpha \eta^{-1}x(t-\tau)=-\alpha \eta^{-1}U(\eta t - \eta \tau)=
-\alpha \eta^{-1}U(\mu - \eta \tau).\]
Če izberemo $\eta := 1/\tau$ in $\beta:=\alpha \tau$, dobimo:
\begin{equation} \label{eq1}
    \begin{split}
 \frac{dU}{d\mu} & =-\beta U(\mu -1).
    \end{split}
\end{equation}
Na množici odvedljivih funkcij, definiramo sledeči linearni operator:
\[L(U)=\frac{dU}{d\mu} + \beta U(\mu -1).\]
Ker obravnavana DDE spominja na NDE, pri katerih so rešitve eksponentne funkcije, poskusimo tudi tukaj
iskati rešitve DDE \eqref{eq1} z nastavkom $U(\mu)=e^{\lambda \mu}$. Ko ga vstavimo v linearni operator,
dobimo:
\[L(e^{\lambda \mu})= \lambda e^{\lambda \mu} + \beta e^{\lambda (\mu -1)}=
e^{\lambda \mu}(\lambda + \beta e^{-\lambda}).\]
Iščemo $\lambda \in \C$ tako, da bo vrednost dobljenega izraza enaka 0. Ker
za vsako kompleksno število $\lambda$ velja $e^{\lambda \mu} \neq 0$ , moramo rešiti naslednjo karakteristično
enačbo:
\[ h(\lambda) \equiv \lambda + \beta e^{-\lambda} =0.\]
Če je $\lambda$ ničla te enačbe, je $e^{\lambda\mu}$ rešitev DDE $(1)$.
Nastavek $\lambda = a + ib$ vstavimo v enačbo:
\begin{equation*}
    \begin{split}
        a + ib + \beta e^{-(a+ib)} &= 0 \\
        a + ib + \beta e^{-a}e^{-ib} &= 0 \\
        a + ib + \beta e^{-a}(\cos(b) + \sin(-b)) &= 0 \\
        a + \beta e^{-a}\cos(b) + i(b-\beta e^{-a}\sin(b)) &= 0
    \end{split}      
\end{equation*}
Dobimo sistem enačb za $a$ in $b$:
\begin{equation*}
    \begin{split}
        a &= -\beta e^{-a}\cos(b) \\
        b &= \beta e^{-a}\sin(b).
    \end{split}      
\end{equation*}
%Spomnimo se: Za $n$-krat zvezno odvedljivo funkcijo $h$ je $\lambda \in \C$ ničla reda $n$, če velja:
%\[ h(\lambda) = h'(\lambda) = \dots = h^{(n-1)}(\lambda)=0, \quad h^{(n)}(\lambda)\neq 0.\]

\begin{trditev}
    Števila $\mu^{j}e^{\lambda \mu}, j=0,1,\dots,n$, so rešitve enačbe \eqref{eq1} natanko
    tedaj, ko je $\lambda$ $(n+1)$-kratna ničla karakteristične funkcije $h$.
\end{trditev}

\begin{proof}[Dokaz]
    Linearni operator $L$ $n$-krat odvajamo po $\lambda$. Ker $k$-ti odvod komutira z $L$, dobimo:
    \[L(\mu^{k}e^{\lambda\mu})=\frac{\partial^k}{\partial\lambda^k}(e^{\lambda\mu}h(\lambda))
    =e^{\lambda\mu}\left(\sum_{k=0}^{n}\binom{n}{k}h^{(k)}(\lambda)\mu^{n-k}\right). \]
    To je enako 0 natanko tedaj, ko je $\lambda$ $(n+1)$-kratna ničla funkcije $h$.
\end{proof}

\noindent Funkcija $h$ je analitična, zato zanjo veljajo naslednje lastnosti:
\begin{enumerate}
    \item[(A)] Za $\forall R > 0$, je množica ničel, ki zadoščajo pogoju $|\lambda|\leq R$ končna. Torej je
            množica ničel funkcije $h$ števna množica.
    \item[(B)] Naj bo množica ničel neskončna (označimo jo z $\{\lambda_{n}\}_{n=1}^{\infty}$), potem velja
    $|\lambda_{n}| \to \infty$. Ker je $|\beta|e^{-\text{Re}(\lambda_{n})}=|\lambda_n|$, sledi, da
    $\text{Re}(\lambda_n) \to -\infty$. Prav tako za vsak $a \in \R$ obstaja končno mnogo ničel $\lambda_n$,
    da je $\text{Re}(\lambda_n) \geq a$. 
    \item[(C)] Če je $\lambda$ ničla funkcije $h$, je njen red končen.
    \item[(D)] Če je $\lambda$ ničla funkcije $h$, je tudi $\overline{\lambda}$ ničla.
\end{enumerate}


\begin{trditev}
    Za karakteristično funkcijo $h$ velja naslednje:
    \begin{enumerate}
        \item Če je $\beta < 0$, potem obstaja natanko ena realna ničla $\lambda$ in je pozitivna.
        \item Če je $0 < \beta < 1/e$, potem obstajata natanko dve različni negativni realni
         ničli, $\lambda_{1} < \lambda_{2}$, za kateri velja:
         \[ \lim_{\beta \to 0}\lambda_{1}=-\infty \quad \text{in} \quad \lim_{\beta \to 0}\lambda_{2}=0. \]
        \item Če je $\beta = 1/e$, potem je edina realna ničla $\lambda = -1$ in je dvakratna.
        \item Če je $\beta > 1/e$, potem funkcija nima realnih ničel.
    \end{enumerate}
\end{trditev}

\begin{proof}[Dokaz]
    Trditev se v celoti osredotoča na primer, ko je $\lambda\in\R$ zato bo to tudi naš privzetek.
    \begin{enumerate}
        \item Funkcija $h$ je naraščajoča vzdolž realne osi in velja $h \to \pm \infty$, ko $\lambda \to \pm \infty$. Obstaja 
        torej natanko ena realna ničla. Ker je $\beta < 0$, zanjo velja $\lambda = -\beta e^{-\lambda} >0$.
        \item Funkcija $h$ ima pri $\lambda=\ln\beta$ stacionarno točko z negativno lastno vrednostjo. Po drugi strani pa velja, da je gre $h\to\infty$ za $\lambda\to\pm\infty$
        , zato obstajata natanko dve realni ničli. Ker je 
        $\beta > 0$, za eno realno ničlo velja $\lambda = -\beta e^{-\lambda} <0$. Drugi del trditve sledi iz dejstva, da velja:
        $ \beta = -\lambda e^{\lambda} \to 0$ natanko tedaj, ko gre $\lambda \to 0$ ali $\lambda \to -\infty$.
        \item Kot v prejšnjem primeru imamo tudi tokrat stacionarno točko, ki pa je hkrati enaka ničli. $\lambda = -1$. Velja $h'(-1)=0$ in $h''(-1)=1$, torej
        je ničla reda 2.
        \item Stacionarna točka ima v tem primeru pozitivno vrednost. Ker je na intervalu pred njo funkcija padajoča,
         za njo pa padajoča, realne ničle ne obstajajo. 
    \end{enumerate}
\end{proof}

Za $\beta > 0$ ima enačba lepe lastnosti, kar nam pove naslednja trditev.

\begin{trditev}
    Za karakteristično funkcijo $h$ velja:
    \begin{enumerate}
        \item Če je $0<\beta < \pi /2$, potem obstaja $\delta > 0$, da je Re($\lambda$) $\leq -\delta$
        za vse ničle $\lambda$.
        \item Če je $\beta = \pi /2$, potem sta $\lambda_{1,2}=\pm i\pi /2$ ničli reda 1.
        \item Če je $\beta > \pi /2$, potem so ničle oblike $\lambda = a + ib$, kjer je $a>0$ in 
         $b \in (\pi /2, \pi)$.
    \end{enumerate} 
\end{trditev}

\begin{proof}[Dokaz]
(1)  Iz $\beta > 0$ sledi: Če obstaja $\lambda = a +ib$ z $a \geq 0$ in $b > 0$, ki reši sistem za $a$ in $b$,
je $\cos(b) \leq 0 < \sin(b)$. Torej mora biti $b \in S = \bigcup_{n=0}^{\infty}\{[\pi / 2,\pi)+2n\pi\}.$ Velja
tudi $\frac{\sin(b)}{b}=\frac{e^{a}}{\beta}$. Ker je tudi $\frac{d}{db}\frac{\sin(b)}{b} < 0$ za $b \in S$
in $\frac{\sin(\pi /2)}{\pi / 2} = \frac{2}{\pi}$, je torej $\frac{\sin(b)}{b} \leq \frac{2}{\pi} \quad
\forall b \in S$.
Dobili smo naslednjo neenakost: 
\[\frac{1}{\beta} \leq \frac{e^a}{\beta} = \frac{\sin(b)}{b} \leq \frac{2}{\pi}.\]
Sledi, da je $\beta \geq \pi/2$. Torej, če je $\beta < \pi/2$, je potem Re($\lambda$) $<0$ za vsako ničlo
$\lambda$. Ko upoštevamo še lastnost (B) funckije $h$, je prvi del trditve dokazan. \\
(2)  Drugi del trditve se dokaže z direktnim izračunom. \\
(3)  Za dokaz tretjega dela nastavimo $\lambda = re^{i\theta}$ in vstavimo v karakteristično enačbo, pri čemer
upoštevamo Eulerjevo formulo:
\[ r(\cos(\theta - \pi)+i\sin(\theta - \pi))=\beta e^{-a}(\cos(-b)+i\sin(-b)).\]
Veljati mora torej:
\[r=\beta e^{-a} \quad \text{in} \quad \theta - \pi = -b + 2k\pi, \quad k \in \Z.\]
Rešitve enačbe iščemo v prvem kvadrantu, torej $a(\theta)>0,\quad b(\theta) = \pi - \theta > 0, \quad
k=0$ in $\theta \in (0,\pi/2)$. Ko upoštevamo še $y/x = \tan(\theta)$, dobimo naslednjo družino rešitev:
\begin{equation*}
    \begin{split}
        x(\theta) &= (\pi - \theta)\cot(\theta) \\
        y(\theta) &= \pi - \theta, \quad 0<\theta <\pi/2 \\
        \beta(\theta) &= \frac{\pi - \theta}{\sin(\theta)}e^{a(\theta)}. 
    \end{split}      
\end{equation*}
$a, b$, in $\beta$ so zvezne funkcije spremenljivke $\theta$ na intervalu $(0,\pi/2)$. Ko $\theta \to 0$,
velja:
\[ a(\theta) \to \infty, \quad b(\theta) \to \pi, \quad \beta(\theta) \to \infty.\]
Za $\theta \to \pi/2$ pa je:
\[a(\theta) \to 0, \quad b(\theta) \to \pi/2, \quad \beta(\theta) \to \pi/2.\]
Ker je $\beta$ strogo padajoča na $(0,\pi/2)$, sledi da je njena zaloga vrednosti $(\pi/2, \infty)$.
\end{proof}

\begin{definicija}
    Dana je DDE $\dot{x}(t)=f(t,x(t-\tau))$ in njena rešitev $\phi(t)$.
    Pravimo, da je $\phi$ stabilna rešitev, če 
    za vsak $\varepsilon > 0$ obstaja $\delta > 0$ tako da za poljubno drugo rešitev $\psi$ z lastnostjo
    $|\psi(t_0)-\phi(t_0)| < \delta$, velja $|\psi(t)-\phi(t)| < \varepsilon$ za $t \geq t_0$.
    Če dodatno velja
    še $\lim_{x \to \infty}|\psi(t)-\phi(t)|=0$, pravimo, da je rešitev asimptotsko stabilna.
\end{definicija}


\begin{trditev}
    Za DDE oblike $\dot{x}(t)= \alpha x(t-\tau), \tau >0, \alpha \in \R$, velja:
    \begin{enumerate}
        \item Če je $\alpha > 0$, je $x=0$ nestabilna rešitev.
        \item Če je $0>\alpha \tau >-\pi/2$, je $x=0$ asimptotsko stabilna.
        \item Če je $\alpha \tau = -\pi/2$, sta rešitvi $x=\sin(\pi\mu/2)$ in $x=\cos(\pi\mu/2)$.
        \item Če je $\alpha\tau < -\pi/2$, je $x=0$ nestabilna.
    \end{enumerate}
\end{trditev}

\noindent \textit{Opomba: }Dokaz sledi direktno iz prejšnjih trditev in rezultata, ki pove, da je $x=0$ asimptotsko stabilna, če 
je $\text{Re}(\lambda) < 0$ za vse $\lambda$, ki so ničle karakteristične funkcije, oziroma nestabilna,
če obstaja ničla $\lambda$ s pozitivnim realnim delom.

\begin{lema}
    Za DDE oblike $\dot{x}(t)= \alpha x(t-\tau), \tau >0, \alpha \in \R$ sta naslednji trditvi
    ekvivalentni:
    \begin{itemize}
        \item Vse rešitve so oscilirajoče.
        \item $\alpha\tau < 1/e$
    \end{itemize}
\end{lema}

\begin{proof}
    Za vsak $\lambda \in \R$ je $U(\mu)=e^{\lambda\mu}$ bodisi monotona, bodisi konstantna funkcija.
    Za realne $\lambda$ torej rešitev $x(t)$ ne oscilira. Funkcija $x(t)$ je torej oscilirajoča natanko tedaj, ko
    je $\lambda\in\mathbb{C}\setminus\mathbb{R}$. Po trditvi 5.3 je to res natanko tedaj, ko je 
    $\beta = \alpha\tau < 1/e$.
\end{proof}

\section{Laplaceova transformacija}

V tem poglavju si bomo ogledali, kako si lahko pri reševanju DDE pomagamo z Laplaceovo transformacijo.
Da bi razumeli postopek, najprej ponovimo in predstavimo nekaj lastnosti omenjene transformacije.

\begin{definicija}
    Naj bo $f:[0,\infty) \to \R$ zvezna funkcija. Laplaceova transformacija funkcije $f$ je kompleksna funkcija
    $F$, podana s predpisom
    \[ F(s)=\int_{0}^{\infty}f(t)e^{-st}dt.\]
\end{definicija}

\noindent V zgornji definiciji funkcije $F$ nismo podali njenega definicijskega območja, zato se je
najprej potrebno vprašati, za katere $s\in\C$, je funkcija $F$ sploh definirana. Predpostavimo, da je
$|f(t)|\leq e^{at}$ za $a\in\R$ in $t\geq0$. Ker za $s\in\C$ velja $|e^{s}|=e^{\text{Re}(s)}$, imamo
\[|e^{-st}f(t)| \leq ce^{-t\text{Re}(s)}e^{at}=ce^{-t(\text{Re}(s)-a)}.\]
Če funkcijo na desni strani te neenačbe integriramo po $t\in [0,\infty)$, dobimo integral, ki je konvergenten za $\text{Re}(s)>a$.
Laplaceova transformacija je torej preslikava, ki funkciji, definirani na $[0,\infty)$, priredi
kompleksno funkcijo, definirano na polravnini $\{s\in\C | \text{Re}(s)>a\}$, za ustrezen $a$.

\begin{zgled}
    Izračunajmo Laplaceovo transformacijo funkcije $f(t)=t$:
    \[F(s)=\int_0^{\infty}te^{-st}dt = \frac{-ste^{-st}-e^{-st}}{s^2}\Bigr|_{0}^{\infty}=\frac{1}{s^2}.\]
    Dobljena funkcija je definirana za vse $s\in\C$ s pozitivnim realnim delom.
\end{zgled}

\subsection{Inverzna transformacija}
Izkazalo se bo, da je za uporabo Laplaceove transformacije pri reševanju DDE ključno, 
ali znamo poiskati tudi njen inverz.
Recimo, da imamo podano funkcijo
$F$ in želimo poiskati funkcijo $f$, katere Laplaceova transformacija je enaka $F$.
Zanimalo nas bo, kako to storiti in katerim pogojem mora zadoščati funkcija $f$.

Recimo, da je integral 
\[\int_0^{\infty}f(r)e^{-sr}dr\]
konvergenten za $\text{Re}(s)>\alpha$. Potem za $a>\alpha$ označimo
\[F(a+it)=\int_0^{\infty}f(r)e^{-(a+it)r}dr.\]
Obe strani pomnožimo z $e^{u(a+it)}$ in integriramo na $[-T,T]$ po spremenljivki $t$:
\[\int_{-T}^{T}e^{u(a+it)}F(a+it)dt=e^{au}\int_0^{\infty}f(r)e^{-ar}\left(\int_{-T}^{T}e^{iut-irt}dt\right)dr.\]
Zaradi absolutne konvergence smo lahko na desni strani zamenjali vrstni red integriranja.
Ko desno stran poenostavimo, dobimo:
\begin{equation} \label{eq2}
    \begin{split}
        \int_{-T}^{T}e^{u(a+it)}F(a+it)dt=2e^{au}\int_0^{\infty}f(r)e^{-ar}\frac{\sin(T(u-r))}{u-r}dr.
    \end{split}
\end{equation}
Funkciji $$k(u,r,T)=\frac{\sin(T(u-r))}{(u-r)}$$ pravimo Dirichlet-Kernelova funkcija in je velikega pomena 
pri iskanju inverzne funkcije. Da bo $k$ zvezna, dodajmo pogoj $k(u,u,T)=T$. % Izkaže se, da je za velike $T$ na
%območju $u=r$ vrednost integrala na desni vedno bolj odvisna le od izraza $f(r)e^{-ar}$. Tako dobimo
%idejo za iskanje funkcije $f$.

Izpeljimo sedaj formulo za reševanje enačbe (2). Integracijski interval razbijemo na tri 
dele: $$(0,\infty) = (0,u-d] \cup (u-d,u+d] \cup (u+d,\infty]$$ za nek $d>0$. Zaradi aditivnosti določenega
integrala dobimo:
\begin{equation*}
    \begin{split}
        \int_{-T}^{T}e^{u(a+it)}F(a+it)dt) &= 2e^{au}\int_0^{u-d}g(u,r,T)dr +
        2e^{au}\int_{u-d}^{u+d}g(u,r,T)dr+ \\
        &+ 2e^{au}\int_{u+d}^{\infty}g(u,r,T)dr,
    \end{split}
\end{equation*}
kjer je $$g(u,r,T)=f(r)e^{-ar}\frac{\sin(T(u-r))}{(u-r)}.$$ Pokazati želimo, da se prvi in tretji integral
na desni strani enačbe bližata ničelni vrednosti, ko gre $T\to\infty$. Pri tem uporabimo naslednjo lemo.
\begin{lema}[Riemann-Lebesgueova lema]
    Če je funkcija $g$ absolutno integrabilna na realni osi, potem je:
    \[\lim_{T\to\infty}\int_{-\infty}^{\infty}g(t)\sin(Tt)dt=0.\]
\end{lema}
\noindent Dokaz leme lahko bralec najde v viru \cite{dokazleme}.

Ker smo predpostavili, da je $\int_0^{\infty}|f(t)|e^{-as}dt<\infty$, je torej funkcija 
$f(r)e^{-ar}(u-r)$ absolutno integrabilna na vsakem zaprtem intervalu, ki ne vsebuje točke $r=u$. Po 
Riemann-Lebesgueovi lemi se lahko znebimo prvega in tretjega integrala na desni strani enačbe. Ostane
nam torej le še:
\[I = 2e^{au}\int_{u-d}^{u+d}f(r)e^{-ar}\frac{\sin(T(u-r))}{u-r}dr.\]
Predpostavimo, da obstaja Taylorjev polinom druge stopnje za funkcijo $f$ na okolici točke $u$.
Zapišimo:
\[f(r)e^{-ar}=f(u)e^{-au}+h(u,r)(u-r),\]
kjer je $|h(u,r)|\leq k, \quad r\in(u-d,u+d).$
Vstavimo v $I$ in dobimo:
\[I = 2f(u)\int_{u-d}^{u+d}\frac{\sin(T(u-r))}{u-r}dr+2e^{au}\int_{u-d}^{u+d}h(u,r)\sin(T(u-r))dr.\]
Ker je $|\sin(T(u-r))|\leq1$, je drugi integral reda $\mathcal{O}(d)$, ko gre $T\to\infty$. Z uvedbo 
nove spremenljivke $v=T(u-r)$, dobimo
\[I=2f(u)\int_{-Td}^{Td}\frac{\sin v}{v}dv +\mathcal{O}(d),\]
oziroma
\[\lim_{T\to\infty}I=2f(u)\int_{-\infty}^{\infty}\frac{\sin v}{v}dv +\mathcal{O}(d).\]
Vrednost zgornjega integrala je enaka $\pi$, zato za vsak $d>0$ velja:
\[\lim_{T\to\infty}I=2\pi f(u)+\mathcal{O}(d).\]
Ker je $d$ poljuben, lahko trdimo, da velja:
\[\lim_{T\to\infty}\int_{-\infty}^{\infty}e^{(a+it)u}F(a+it)dt=2\pi f(u).\]
S tem smo predstavili glavne korake za dokaz izreka, ki je naveden spodaj, pri čemer pa 
smo predpostavili, da je $f$ dvakrat zvezno odvedljiva oz. da lahko poiščemo njen Taylorjev polinom drugega
reda v vsaki točki. 
Z nekaj dodatne analize lahko dokažemo tudi splošnejšo verzijo, ki je navedena spodaj. Dokaz najdemo v viru \cite{knjiga}.

\begin{izrek}
    Naj bo $f$ zvezna funkcija z naslednjimi lastnostmi:
    \begin{enumerate}
        \item[(a)]Integral $\int_0^{\infty}f(t)e^{-at}dt$ je absolutno konvergenten za nek $a>0$,
        \item[(b)]$f$ ima omejen odvod v okolici točke $u>0$.  
    \end{enumerate}
    Potem obstaja
    \[F(s)=\int_0^{\infty}e^{-st}f(t)dt\]
    za $\text{Re}(s)\geq a$. Za $b>a$ pa velja:
    \[\lim_{T\to\infty}\frac{1}{2\pi}\int_{-T}^{T}e^{(b+it)u}F(b+it)dt=f(u).\]
\end{izrek}


S tem smo dobili formulo za izračun inverzne funkcije pri Laplaceovi transformaciji.
Če vpeljemo novo spremenljivko $s=b+ir$ za $r\in[-T,T]$, lahko formulo še malo polepšamo:
\[\frac{1}{2\pi}\int_{-T}^{T}e^{(b+it)u}F(b+it)dr=\frac{1}{2\pi i}\int_{b-iT}^{b+iT}F(s)e^{st}ds.\]
Desni integral je krivuljni interal po daljici med točkama $b-iT$ in $b+iT$. Označimo
\[ \int_{(b)}F(s)e^{st}ds=\lim_{T\to\infty}\frac{1}{2\pi i}\int_{b-iT}^{b+iT}F(s)e^{st}ds,\]
kadar desna stran obstaja.
Končna formula za inverzno funkcijo je torej:
\[f(t)=\int_{(b)}F(s)e^{st}ds.\]

\subsection{Reševanje DDE z Laplaceovo transformacijo}
    Z Laplaceovo transformacijo bomo ponovno rešili DDE iz 3. poglavja, le da imamo tokrat
    podan bolj splošen začetni pogoj:
    \begin{equation} \label{eq3}
        \begin{split}
            \dot{x}(t)=x(t-1), \quad x(t)=\varphi(t), \quad t\in[-1,0].
        \end{split}
    \end{equation}
    Obe strani pomnožimo z $e^{-st}$ in integriramo na intervalu $[0,\infty)$:
     \[\int_{0}^{\infty}\dot{x}(t)e^{-st}dt = \int_{0}^{\infty}x(t-1)e^{-st}dt.\]
    Z uvedbo nove spremenljivke $u=t-1$ preoblikujemo desno stran enačbe:
    \begin{equation*}
        \begin{split}
            \int_{0}^{\infty}x(t-1)e^{-st}dt &= \int_{-1}^{\infty}x(u)e^{-s(u+1)}du \\
             &= \int_{-1}^{\infty}x(t)e^{-s(t+1)}dt \\
             &= e^{-s}\left(\int_{0}^{\infty}x(t)e^{-st}dt + \int_{-1}^{0}x(t)e^{-st}dt\right). 
        \end{split}      
    \end{equation*}
    Na levi strani enačbe uporabimo integracijo per partes. Pri tem predpostavljamo, da za 
    rešitev velja $x(t)e^{-st} \to 0$, ko gre $t \to\infty$.
    \[\int_{0}^{\infty}\dot{x}(t)e^{-st}dt=-x(0)e^{-s}+s\int_{0}^{\infty}x(t)e^{-st}dt.\]
    Sedaj upoštevajmo še, da za $t\in[-1,0]$ velja $x(t)=\varphi(t)$. Enačbo (3) smo pretvorili na naslednjo obliko:
    \[(s-e^{-s})\int_{0}^{\infty}x(t)e^{-st}dt=\varphi(0)e^{-s}+e^{-s}\int_{-1}^0\varphi(t)e^{-st}dt.\]
    Za $s-e^{-s}\neq 0$ dobimo torej:
    \[\int_{0}^{\infty}x(t)e^{-st}dt=\frac{\varphi(0)e^{-s}+e^{-s}\int_{-1}^0\varphi(t)e^{-st}dt}{s-e^{-s}}.\]
    Izraz na desni je zvezna funkcija spremenljivke $s$. Označimo jo z $G(s)$. Po izreku o inverzni
    transformaciji dobimo:
    \[x(t)=\int_{(c)}G(s)e^{st}ds.\]
    Za razliko od 3. poglavja, nam tu rešitve ni uspelo
    izraziti z elementarnimi funkcijami. Kljub temu bi lahko z dodatno analizo obravnavali njene lastnosti.

\section{Linearne diferencialne enačbe z zamikom}

\begin{definicija}
    Linearna DDE je podana na sledeči način:
    \begin{equation} \label{eq4}
        \begin{split}
            \dot{x}(t)=ax(t)=bx(t-\tau)+f(t),\quad a,b\in\R, \quad \tau>0.
        \end{split}
    \end{equation}
    Če je $f(t)=0$, je enačba homogena, sicer je nehomogena. 
\end{definicija}

V tem poglavju bomo obravnavali homogene linearne DDE. Za njihovo reševanje uporabimo nastavek
$x(t)=e^{\lambda t}c$, $\lambda \in\C$, $c\in\R\setminus\{0\}$, in ga vstavimo v enačbo:
\begin{equation*}
    \begin{split}
        h(\lambda)&=\lambda e^{\lambda t}c - ae^{\lambda t}c -be^{\lambda(t-\tau)} \\
         &= \lambda - a -be^{-\lambda\tau}.
    \end{split}      
\end{equation*}
Iščemo ničle karakteristične funkcije $h$. Izraz lahko z uvedbo novih spremenljivk še dodatno 
poenostavimo. Če nastavimo $z:=\lambda\tau$, $\alpha:=a\tau$ in $\beta:=b\tau$, se karakteristična
enačba prevede na
\[h(z)=z-\alpha-\beta e^{-z}=0.\]
Reševanje te karakteristične enačbe je podobno kot pri modelu eksponentne rasti z zamikom.
\begin{trditev}
    Za DDE oblike \eqref{eq4} in rešitev $x=0$ velja:
    \begin{itemize}
        \item Če je $a+b>0$, potem je $x=0$ stabilna.
        \item Če je $a+b<0$ in $b\geq a$, potem je $x=0$ asimptotsko stabilna.
        \item Če je $a+b<0$ in $b<a$, potem obstaja $\tau^{*}>0$, tako da je $x=0$ 
        asimptotsko stabilna za $0<\tau<\tau^{*}$ in nestabilna za $\tau>\tau^{*}$.
    \end{itemize}
\end{trditev}



\section{Linearni sistemi DDE}

\begin{definicija}
    Sistem linearnih DDE s konstantnimi koeficienti je podan z 
    \begin{equation} \label{eq5}
        \begin{split}
            \dot{x}(t)=Ax(t)+Bx(t-\tau)+f(t)\quad \text{za}\quad t\geq0,
        \end{split}
    \end{equation} kjer so
    $A,B\in\R^{n\times n}$, $\tau>0$ in $f$ zvezna funkcija. Začetni pogoj je znova podan s predpisom
    \begin{equation} \label{eq6}
        \begin{split}
            x(t)=\phi(t)\quad \text{za}\quad t\in [-\tau,0],
        \end{split}
    \end{equation}
    kjer je $\phi$ zvezna funkcija. Če je $f(t)\equiv0$, pravimo,
    da je sistem homogen.
\end{definicija}
Za analizo linearnih sistemov DDE bomo uporabili Laplaceovo transformacijo. Najprej jo uporabimo na levi strani sistema:
\begin{equation*}
    \begin{split}
        G_1(s)&= \int_0^{\infty}e^{-st}\dot{x}(t)dt \\
            &= (e^{-st}x(t))\Bigr|_{0}^{\infty} + s\int_0^{\infty}e^{-st}x(t)dt \\
            &= -\phi(0) + sX(s),
    \end{split}      
\end{equation*}
kjer je $X(s)$ Laplaceova transformacija iskane funkcije.
Ko transformiramo še desno stran, dobimo:
\begin{equation*}
    \begin{split}
        G_2(s)&= \int_0^{\infty}e^{-st}\left(Ax(t)+Bx(t-\tau)+f(t)\right)dt \\
            &= A\int_0^{\infty}e^{-st}x(t)dt+B\int_0^{\infty}e^{-st}x(t-\tau)dt+\int_0^{\infty}e^{-st}f(t)dt \\
            &= AX(s) + B\left(\int_0^{\tau}e^{-st}\phi(t-\tau)dt+\int_{\tau}^{\infty}e^{-st}x(t-\tau)dt\right) + F(s) \\
            &= AX(s) + B\left(\int_0^{\tau}e^{-st}\phi(t-\tau)dt+\int_{0}^{\infty}e^{-s(t+\tau)}x(t)dt\right) + F(s) \\
            &= AX(s) + B\left(\int_0^{\tau}e^{-st}\phi(t-\tau)dt+e^{-st}X(s)\right)+F(s) \\
            &= (A+e^{-st}B)X(s) + B\int_0^{\tau}e^{-st}\phi(t-\tau)dt+F(s) \\
            &= (A+e^{-st}B)X(s) + B\Phi(s)+F(s).
    \end{split}      
\end{equation*}
Pri izračunu smo predpostavili, da je funkcija $\phi$ ničelna na $(0,\infty)$. Funkcija $\Phi$ pa 
označuje Laplaceovo transformacijo funkcije $\phi(t-\tau)$.

Ko izenačimo $G_1$ in $G_2$, dobimo:
\[X(s)=K(s)(\phi(0)+B\Phi(s)+F(s)),\]
kjer je 
\[K(s)=(sI-A-e^{-st}B)^{-1}.\]
Da bi dobili lepo rešitev enačbe, moramo poiskati inverzno transformacijo funkcije $K$, ki jo bomo označili s $k$.
Spomnimo, da mora taka funkcija rešiti sistem \eqref{eq5} za $f=0$ in zadostiti začetnemu pogoju
\[
        \xi(\theta) =
        \begin{dcases}
            I &\quad \text{za  } \theta=0 \\
            0 &\quad \text{za  } \theta \in [-\tau,0). \\
        \end{dcases}
    \]
Kljub nezveznosti funkcije $\xi$ pri $\theta=0$ funkcija $k$ obstaja za $t\geq0$. Pravimo ji fundamentalna matrična
rešitev sistema \eqref{eq5}.

Naslednji izrek za bolj splošne sisteme DDE nam bo podal pomemben rezultat o enoličnosti in obstoju rešitve.

\begin{izrek}
    Podan je linearen sistem DDE oblike:
    \begin{equation}\sum_{i=0}^m\left(A_i\dot{x}(t-\tau_i)+B_ix(t-\tau_i)\right)=f(t),\end{equation}
    kjer je 
    \[0=\tau_0<\tau_1<\dots<\tau_m, \quad \det(A)\neq0,\]
    pri začetnem pogoju
    \[x(t)=\phi(t), \quad t \in[0,\tau_m].\]
    Z $S$ označimo množico točk oblike $t=\sum_{i=0}^{m}j_i\tau_i$, kjer so $j_i\in\Z$. Naj bo še $S_1=S\cap[\tau_m,m\infty)$ in
    $S_2=S\cap(\tau_m,m\infty)$. Naj bo vektor $\phi$ reda $\mathcal{C^1}([0,\tau_m])$ in $f$ reda $\mathcal{C^0}([0,\infty)$,
    razen morda za kočno mnogo točk nezveznosti na $S_1$. Potem obstaja natanko ena funkcija $x(t)$, ki je zvezna 
    za $t\geq0$, zadošča začetnemu pogoju in reši sistem (7) za $t\geq\tau_m$, $t\notin S_2$. Nadalje, $x(t)$ je reda
    $\mathcal{C^1}$ za $t>0$, $t\notin S_1$. Če je $f$ zvezna na $[0,\infty)$, je $x(t)$ reda $\mathcal{C^1}$ za vse $t>0$ in 
    reši sistem (7) za vse $t>\tau_m$ natanko tedaj, ko je zvezno odvedljiva v točki $t=\tau_m$. To je natanko tedaj, ko velja
    $\sum_{i=0}^m\left(A_ig'(\tau_m-\tau_i)+B_ig(\tau_m-\tau_i)\right)=f(\tau_m)$.
\end{izrek}

\noindent\textit{Opomba pred dokazom:} Če vzamemo $A_0=I$, $A_i=0$ za vse $i\neq0$, $B_0\neq0$, $B_i\neq0$ za nek $i\neq0$ ter 
$B_j=0$ za $j\neq0,i$, smo sistem prevedli na obliko (5) z zamikom $\tau_i$.

\begin{proof}
    Ker je $A_0$ obrnljiva, lahko sistem (7) pomnožimo z $A_0^{-1}$. Brez škode za splošnost lahko zato predpostavimo, da 
    za sistem (7) velja $A_0=I$. Definirajmo:
    \[g(t)=f(t)-\sum_{i=1}^{m}\left(A_i\dot{x}(t-\tau_i)+B_ix(t-\tau_i)\right).\]
    Ker je $f$ reda $\mathcal{C^0}$ na $[0,\infty)$ razen na $S_1$ in $\phi$ reda $\mathcal{C^1}$ na $[0,\tau_m]$, je
    $g$ reda $\mathcal{C^0}$ na $[\tau_m,\tau_m+\tau_1]$. Po integraciji sledi, da obstaja enolična funkcija $x(t)$, ki zadošča sistemu (7) na 
    $(\tau_m,\tau_m+\tau_1)$ in začetnemu pogoju. Ta funkcija je reda $\mathcal{C^1}$ na $[0,\tau_m+\tau_1]$ razen morda v 
    točki $t=\tau_m$. Sledi, da je 
    $g$ reda $\mathcal{C^0}$ na $[\tau_m,\tau_m+2\tau_1]$ razen morda v točkah $t=\tau_m+\tau_j$, $j=1,\dots,m$, $t<\tau_m+2\tau_1$.
    
    Z integracijo 
    razširimo rešitev na $[0,\tau_m+2\tau_1]$. 
    Tu je rešitev reda $\mathcal{C^0}$, zadošča sistemu na $(\tau_m,\tau_m+2\tau_1)$
    razen morda v točkah $t=\tau_m+\tau_j$, $j=1,\dots,m$, $t<\tau_m+2\tau_1$ in reda $\mathcal{C^1}$ na $[0,\tau_m+2\tau_1]$, razen 
    morda v točkah $t=\tau_m+\tau_j$, $j=0,\dots,m$, $t<\tau_m+2\tau_1$.
    Postopek lahko induktivno ponavljamo v neskočnčnost. Vidimo, da je
    rešitev $x(t)$ reda $\mathcal{C^0}$ na $[0,\infty)$, in reda $\mathcal{C^1}$ razen morda na $S_1$.
    Iz zgronje konstrukcije je lahko videti, da če je $f$ reda $\mathcal{C^0}$ na $[0,\infty)$, je $x(t)$
    reda $\mathcal{C^1}$ na $[0,\infty)$ in reši sistem za vse $t>\tau_m$ natanko tedaj, ko je zvezno odvedljiva v $t=\tau_m$.
    To je res natanko tedaj, ko je $\dot{x}(\tau_m+0)=\phi'(\tau_m-0)$, torej, ko drži pogoj iz izreka.
\end{proof}

\begin{lema}
    Rešitev sistema \eqref{eq5} pri začetnem pogoju \eqref{eq6} lahko izrazimo kot
    \[x(t):=x(t;\phi,f)=x(t;\phi,0)+x(t;0,f).\]
\end{lema}

\begin{proof}
    Za $t\in [0,\tau]$ velja:
    \begin{equation*}
        \begin{split}
            \dot{x}(t;\phi,f) &= Ax(t;\phi,f)+B\phi(t-\tau)+f(t) \\
            \dot{x}(t;\phi,0)  &= Ax(t;\phi,0)+B\phi(t-\tau)\\
            \dot{x}(t;0,f)  &= Ax(t;0,f)+f(t).
        \end{split}      
    \end{equation*}
Za $t\in[0,\tau]$ je torej
\[\dot{x}(t;\phi,0)+\dot{x}(t;0,f)=A\left(x(t;\phi,0)+x(t;0,f)\right)+B\phi(t-\tau)+f(t).\]
Sledi, da je
\[x(t):=x(t;\phi,f)=x(t;\phi,0)+x(t;0,f).\]
Po enoličnosti rešitve je to res za $t\geq0$.
\end{proof}

\begin{zgled}
    Podan je homogen linearen sistem DDE
    \[\dot{x}=Ax(t)+Bx(t-\tau)\quad \text{za}\quad t\geq0,\]
    kjer je $A=
    \begin{bmatrix}
        0 & 1 \\
        -2 & 3 
    \end{bmatrix}$
    in $B=
    \begin{bmatrix}
        0 & 1 \\
        1 & 0 
    \end{bmatrix}.$
    Začetni pogoj je podan kot 
    $\phi(t)=\begin{bmatrix}
        1 \\
        1 
    \end{bmatrix}$
    za $t\in[-\tau,0]$. \\
    Najprej izračunamo lastne vrednosti matrike $A$ in pripadajoče lastne vektorje:
    \[\text{det}(A-\lambda I)=\lambda^2-3\lambda+2=0.\]
    Lastni vrednosti $A$ sta torej $\lambda_1=1$ in $\lambda_2=2$.
    Iz homogenih sistemov
    \[(A-\lambda_1 I)v_1=\begin{bmatrix}
        0 \\
        0 
    \end{bmatrix}
    \quad \text{in} \quad
    (A-\lambda_2 I)v_2=\begin{bmatrix}
        0 \\
        0 
    \end{bmatrix}\]
    dobimo lastna vektorja $v_1=\begin{bmatrix}
        1 \\
        1 
    \end{bmatrix}$
    in  $v_2=\begin{bmatrix}
        1 \\
        2 
    \end{bmatrix}$.
    Rešitve sistema 
    \[\dot{x}(t)=Ax(t)\]
    so torej 
    \[x_h(t)=c_1e^t\begin{bmatrix}
        1 \\
        1 
    \end{bmatrix}
    +c_2e^{2t}\begin{bmatrix}
        1 \\
        2 
    \end{bmatrix}.\]
    za konstanti $c_1$ in $c_2$. 
    Za $t\in[0,\tau]$ je
    \begin{equation*}
        \begin{split}
            \dot{x}(t) &= Ax(t)+B\phi(t-\tau) \\
            &= \begin{bmatrix}
                0 & 1 \\
                -2 & 3 
            \end{bmatrix}x(t)+\begin{bmatrix}
                0 & 1 \\
                1 & 0 
            \end{bmatrix}
            \begin{bmatrix}
                1  \\
                1  
            \end{bmatrix} \\
            &= \begin{bmatrix}
                0 & 1 \\
                -2 & 3 
            \end{bmatrix}x(t)+
            \begin{bmatrix}
                1  \\
                1  
            \end{bmatrix}.
        \end{split}
    \end{equation*}
    Iščemo partikularno rešitev 
    \[x_p(t)=\begin{bmatrix}
        p_1  \\
        p_2  
    \end{bmatrix}\]
    za neki konstantni $p_1$ in $p_2$, ki reši ta sistem. $x_p(t)$ vstavimo v enačbo in dobimo
    \[\dot{x_p}(t)=\begin{bmatrix}
        0  \\
        0  
    \end{bmatrix} = \begin{bmatrix}
        0 & 1 \\
        -2 & 3 
    \end{bmatrix}
    \begin{bmatrix}
        p_1  \\
        p_2  
    \end{bmatrix} +
    \begin{bmatrix}
        1  \\
        1  
    \end{bmatrix} = 
    \begin{bmatrix}
        p_2+1  \\
        -p_1+3p_2+1  
    \end{bmatrix}.\]
    Sledi, da je $p_1=-2$ in $p_2=-1$. Partikularna rešitev je torej
    \[x_p(t)=
    \begin{bmatrix}
        -2  \\
        -1  
    \end{bmatrix}.\]
    Za $t\in[0,\tau]$ je splošna rešitev sistema
    \[x(t)=x_h(t)+x_p(t)= c_1e^t\begin{bmatrix}
        1 \\
        1 
    \end{bmatrix}
    +c_2e^{2t}\begin{bmatrix}
        1 \\
        2 
    \end{bmatrix} + \begin{bmatrix}
        -2  \\
        -1  
    \end{bmatrix}.\]
    Z upoštevanjem začetnega pogoja določimo še konstanti $c_1$ in $c_2$:
    \[x(0)=\begin{bmatrix}
        1  \\
        1  
    \end{bmatrix} = c_1 \begin{bmatrix}
        1  \\
        1  
    \end{bmatrix} + c_2 \begin{bmatrix}
        1  \\
        2  
    \end{bmatrix} + \begin{bmatrix}
        -2  \\
        -1  
    \end{bmatrix} = \begin{bmatrix}
        c_1 + c_2 -2  \\
        c_1+2c_2 -1  
    \end{bmatrix}.\]
    Dobimo, da je $c_1=4$ in $c_2=-1$. Končna rešitev je torej 
    \[x(t)=x_h(t)+x_p(t)= 4e^t\begin{bmatrix}
        1 \\
        1 
    \end{bmatrix}
    -e^{2t}\begin{bmatrix}
        1 \\
        2 
    \end{bmatrix} + \begin{bmatrix}
        -2  \\
        -1  
    \end{bmatrix}, \quad t\in[0,\tau].\]
    Poiščimo še rešitev za $t\in[\tau,2\tau]$. Rešitve sistema $\dot{x}=Ax(t)$ so še vedno oblike
    \[ x_h(t)=c_1e^t\begin{bmatrix}
        1 \\
        1 
    \end{bmatrix}
    +c_2e^{2t}\begin{bmatrix}
        1 \\
        2 
    \end{bmatrix}.\]
    Za $t\in[\tau,2\tau]$ je
    \begin{equation*}
        \begin{split}
            \dot{x}(t) &= Ax(t)+Bx(t-\tau) \\
            &= \begin{bmatrix}
                0 & 1 \\
                -2 & 3 
            \end{bmatrix}x(t)+\begin{bmatrix}
                0 & 1 \\
                1 & 0 
            \end{bmatrix}
            \begin{bmatrix}
                4e^{t}-e^{2t}-2  \\
                4e^{t}-2e^{2t}-1  
            \end{bmatrix} \\
            &= \begin{bmatrix}
                0 & 1 \\
                -2 & 3 
            \end{bmatrix}x(t)+
            \begin{bmatrix}
                4e^{t}-2e^{2t}-1  \\
                4e^{t}-e^{2t}-2  
            \end{bmatrix}.
        \end{split}
    \end{equation*}
    Označimo
    \[\Psi(t)=\begin{bmatrix}
        4e^{t}-e^{2t}-2  \\
        4e^{t}-2e^{2t}-1  
    \end{bmatrix}.]
    Ker bi bilo tu iskanje partikularne rešitve z nastavkom prezahtevno, uporabimo variacijo konstante.
    Matrika Wronskega je enaka
    \[W(t)=\begin{bmatrix}
        e^{\lambda_1t} & e^{\lambda_2t} \\
        (e^{\lambda_1t})' & (e^{\lambda_1t})' 
    \end{bmatrix}
    =\begin{bmatrix}
        e^t & e^{2t} \\
        e^t & 2e^{2t} 
    \end{bmatrix}.\]
    Partikularno rešitev iščemo v obliki $x_p(t)=W(t)\vec{D}(t)$, $\vec{D}(t)=(D_1(t),D_2(t)).$ Ko to vstavimo v enačbo, dobimo 
    \begin{equation*}
        \begin{split}
            \dot{W}\vec{D}+W\dot{\vec{D}} &= AW\vec{D}+B\Psi(t) \\
            \dot{\vec{D}}&=W^{-1}(B\Psi(t)) \\
            \vec{D} &=\int W^{-1}(B\Psi(t))dt
        \end{split}
    \end{equation*}
    Za tak vektor $\vec{D}$ bo torej $x_p(t)=W(t)\vec{D}(t)$ partikularna rešitev sistema.
    Izračunajmo najprej $W^{-1}$:
    \[W^{-1}=\frac{1}{2e^{3t}-e^{3t}}\begin{bmatrix}
        2e^{2t} & -e^{2t} \\
        -e^t & e^t
    \end{bmatrix}=\frac{1}{e^{3t}}\begin{bmatrix}
        2e^{2t} & -e^{2t} \\
        -e^t & e^t
    \end{bmatrix}.\] 
    Po formulu izračunamo še $\vec{D}$:
    \begin{equation*}
        \begin{split}
            \vec{D} &=\int W^{-1}(B\Psi(t))dt \\
                    &= \int W^{-1}(\begin{bmatrix}
                        4e^{t}-2e^{2t}-1  \\
                        4e^{t}-e^{2t}-2  
                    \end{bmatrix})dt \\
                    &= e^{-3t} \int \begin{bmatrix}
                        2e^{2t} & -e^{2t} \\
                        -e^t & e^t
                    \end{bmatrix} \begin{bmatrix}
                        4e^{t}-2e^{2t}-1  \\
                        4e^{t}-e^{2t}-2  
                    \end{bmatrix} dt \\
                    &= e^{-3t}\int \begin{bmatrix}
                        -3e^{4t}+4e^{3t}  \\
                        e^{3t}-e^t  
                    \end{bmatrix} dt \\
                    &= \int \begin{bmatrix}
                        -3e^{4t}+4  \\
                        1-e^{-2t}  
                    \end{bmatrix} dt \\
                    &= \begin{bmatrix}
                        -3e^t+4t  \\
                        t+\frac{1}{2}e^{-2t}  
                    \end{bmatrix}.
        \end{split}
    \end{equation*}
    Partikularna rešitev je torej 
    \begin{equation*}
        \begin{split}
            x_p(t) &= W\vec{D} \\
                    &= \begin{bmatrix}
                        e^t & e^{2t} \\
                        e^t & 2e^{2t} 
                    \end{bmatrix} \begin{bmatrix}
                        -3e^t+4t  \\
                        t+\frac{1}{2}e^{-2t}  
                    \end{bmatrix} \\
                    &= \begin{bmatrix}
                        te^{2t}-3e^{2t}+4te^t+\frac{1}{2}  \\
                        2te^{2t}-3e^{2t}+4te^{t}+1 
                    \end{bmatrix}
        \end{split}
    \end{equation*}
    Za $t\in[\tau,2\tau]$ je splošna rešitev sistema torej
    \[x(t)=x_h(t)+x_p(t)=\begin{bmatrix}
        te^{2t}+(c_2-3)e^{2t}+4te^t+c_1e^t+\frac{1}{2}  \\
        2te^{2t}+(2c_2-3)e^{2t}+4te^{t}+c_1e^t+1 
    \end{bmatrix}.\]
    Za določitev konstant $c_1$ in $c_2$ uporabimo pogoj
    \[x(\tau) =\Psi(\tau)=\begin{bmatrix}
        4e^{\tau}-e^{2\tau}-2  \\
        4e^{\tau}-2e^{2\tau}-1  
    \end{bmatrix} = \begin{bmatrix}
        \taue^{2\tau}+(c_2-3)e^{2\tau}+4te^\tau+c_1e^\tau+\frac{1}{2}  \\
        2\taue^{2\tau}+(2c_2-3)e^{2\tau}+4te^{\tau}+c_1e^t+1 
    \end{bmatrix}. \]
    Podobno kot v tretjem poglavju bi lahko induktivno poiskali rešitve na vseh intervalih oblike
    $[(n-1)\tau,\tau], n\in\N$.

\end{zgled}


\subsection{Reševanje sistemov v splošnem}
V tem podpoglavju bomo izpeljali splošen postopek za reševanje linearnih sistemov DDE oblike 
\[\dot{x}(t)=L(x_t),\]
kjer je $L:C\to\C^n$ za $C=\mathcal{C}([-\tau,0],\C^n)$ linearna preslikava, $x_t$ pa je definiran kot 
\[x_t(\theta):=x(t+\theta) \quad \text{za} \quad -\tau\leq\theta\leq0.\]
Rešitve sistema iščemo z nastavkom 
\[x(t)=e^{\lambda t}v, \quad v\neq0,\]
kjer je $\lambda\in\C$ in  $v\in\C^n$. Uvedimo še novo oznako $\text{exp}_{\lambda}(\theta):=e^{\lambda\theta}$.
Potem velja
\[x_t(\theta)=x(t+\theta)=e^{\lambda(t+\theta)}v=e^{\lambda t}e^{\lambda\theta}v=e^{\lambda t}\text{exp}_{\lambda}(\theta)v.\]
Ker je $L$ linearna preslikava in $e^{\lambda t}\in\C$ za vse $\lambda$ in $t$, dobimo z upoštevanjem nastavka naslednji
pogoj:
\[\dot{x}(t)=\lambda e^{\lambda t}v=L(x_t)=L(e^{\lambda t}(\text{exp}_{\lambda})v)=e^{\lambda t}L((\text{exp}_{\lambda})v).\]
Obe strani delimo z $e^{\lambda t}$ in dobimo 
\[ \lambda v=L((\okr)v).\]
Naj bo $\{e_1,e_2,\dots,e_n\}$ standardna baza prostora $\C^n$. Potem je 
$v=\sum_{i=1}^nv_ie_i$, kjer je $v_i\in\C$ za vse $i$. Iz linearnosti preslikave $L$ sledi 
\[L((\okr)v)=\sum_{i=1}^{n}v_i((\okr)e_i).\]
Definirajmo $n\times n$ matriko $L_\lambda$ na sledeči način:
\[ L_{\lambda}=[L((\okr)e_j)]_{j=1,\dots,n}.\]
$j$-ti stolpec matrike je torej vektor $L((\okr)e_j)$. Potem je
\begin{equation*}
    \begin{split}
        \lambda v &= L((\okr)v) \\
                &= \sum_{j=1}^{n}v_jL((\okr)e_j) \\
                &= (L((\okr)e_1)|(L((\okr)e_2)|\cdots|(L((\okr)e_n)) \cdot (v_1,v_2,\dots,v_n)^{T} \\
                &= L_{\lambda}v.
    \end{split}
\end{equation*}
Ker je torej $L_\lambda v =\lambda v$ in $v\neq0$, je $x(t)=e^{\lambda t}v$ rešitev sistema natanko tedaj, ko je
$\lambda$ lastna vrednost matrike $L_\lambda$, oziroma
\[ \text{det}(L_\lambda-\lambda I)=0,\]
vektor $v$ pa njen lastni vektor.
Poglejmo si znova homogen sistem 
\[ \dot{x}(t)=Ax(t)+Bx(t-\tau),\]
z $A$ in $B$ $n\times n$ matriki in zamik $\tau>0$. Linearno preslikavo $L$ definirajmo na naslednji način:
\[L(y)=Ay(0)+By(-\tau).\]
Matrika $L_\lambda$ je v tem primeru
\begin{equation*}
    \begin{split}
        L_\lambda &= (L_i((\okr)e_j)) \\
                &= ((A\okr(0)e_j+B\okr(-\tau)e_j)_i) \\
                &= ((Ae_j+e^{-\lambda\tau}Be_j)_i) \\
                &= (Ae_1+e^{-\lambda\tau}Be_1|Ae_2+e^{-\lambda\tau}Be_2|\cdots|Ae_n+e^{-\lambda\tau}Be_n) \\
                &= (A+e^{-\lambda\tau}B)(e_1,e_2,\dots,e_n) \\
                &= (A+e^{-\lambda\tau}B)I \\
                &= A+e^{-\lambda\tau}B.
    \end{split}
\end{equation*}
Karakteristična enačba sistema je torej
\[\text{det}(A+e^{-\lambda\tau}B-\lambda I)=0.\]
Rešitve sistema $\dot{x}(t)=Ax(t)+Bx(t-\tau)$ so torej oblike $x(t)=e^{\lambda t}v$, kjer je $\lambda$ lastna vrednost in $v$ pripadajoči lastni 
vektor matrike $A+e^{-\lambda\tau}B$. Opazimo, da je reševanje linearnih sistemov DDE precej bolj zahtevno kot
linearnih reševanje sistemov NDE, saj tu karakteristična funkcija ni več polinom.

\section{Viri in literatura}

\overfullrule=0mm\printbibliography
\end{document}
